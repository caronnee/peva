\documentclass[a4paper,11pt,final]{report}
\usepackage[utf8]{inputenc}
\usepackage[slovak]{babel}
\usepackage{amsmath}
\usepackage{amssymb}
\usepackage{amsthm}
\begin{document}
\begin{titlepage}
\begin{center}
\vspace{1.5in}
{\rm Univerzita Karlova v Prahe\\
    Matematicko-fyzikálna fakulta}
\par
\vspace{0.7in}
{\huge \bf Bakalárska práca}
\par
\vspace{0.5in}
{Eva Pešková}
\par
\vspace{0.5in}
Codewars, vojna robotov
\par
\vfill
Katedra softwarového inžinierstva
\par
\vspace{0.5in}
Vedúci bakalárskej práce: Mgr. Tomáš Poch
\par
\vspace{0.5in}
Všeobecná informatika
\par
\vspace{0.5in}
2009
\end{center}
\end{titlepage}
\vfill
Prehlasujem, že som svoju bakalársku prácu napísala samostatne a \\
výhradne s použitím citovaných prameňov. \\
Súhlasím so zapožičiavaním práce.\\
\par
V Prahe \today
\begin{flushright}
Eva Pešková
\end{flushright}
\tableofcontents
\chapter{Uvod}% na com sa to skusa, preco, co je hlanym  cielom, apod.
\chapter{Analýza}
\section{Codewars a jeho alternativy}
\subsection {Robowars}
\subsection {Pogamut}
\section{Problem optimalnej strategie viacerych hracov}
\section{Prakticke vyuzitie}
\chapter{Implementácia prostredia}
\section{Hracia plocha} % z coho sa sklada
\subsection{Generovanie map}% ako sa generuje, podpora ukladania, zadne zhluky, vyhody, nevyhody, 
\subsection{Vlastnosti stien} % vseobecne oba rozne vlastnosti, ake maju steny, k comu to je
\section{Jazyk hry a priebeh hry} %bison, preco
Po načítaní všetkých robotov, ich programov sa hra klada z trojice (FrontaUdalosti, Mapa, ŽiviBoti). Samotná hra prebieha tak, že pokiaľ existujú živí boti, spracuje sa nsledujúca udalost z FrontyUdalosti. O zaradenie zpať a prípadné vymazanie sa v prípade úmrtia sa objekty starajú sami.\\
Objekty = steny, boti, strely. \\
Stav robota vzhladom na svet charakterizuje šestica: (InstruktionStack,Instructionpointer, Hitpoints, X, Y, natocenie)\\
Jednotlivé inštrukcie sa prejavia takto (zatial nepredpokladame koliziu):
\begin{table}[ht]
\caption{Vplyv robota na svet}   % title of Table
\centering                          % used for centering table
\begin{tabular}{|p{3cm}|p{3cm}|p{4cm}|p{4cm}|}        
\hline\hline                        %inserts double horizontal lines
Instrukcia & parameter & stav robota po instrukcii & Stav sveta po instrukcii\\    % inserts table
%heading
\hline                              % inserts single horizontal line
Viditelne instrukcie vzhladom na svet\\\hline
step & Z[4] alebo typy FORWARD, BACKWARD, LEFT, RIGTH & (IS, IP+1, HP, x+(0,1),y+(0,1), natocenie) & ( Fronta.pop().insert(bot), Mapa, ZiviBoti)\\\hline
step & -(default parameter = FORWARD) & rovnako & rovnako\\\hline %run nevediem! to iste ako preplanovanie za menej casu
turnLeft & - & (InstruktionStack, Instructionpointer+1, Hitpoints, X, Y, natocenie+1)&(FrontaUdalosti.pop(). insert(bot), Mapa, ŽiviBoti)\\ \hline
turnRight  & - & (InstruktionStack, Instructionpointer+1, Hitpoints, X, Y, natocenie-1)&(FrontaUdalosti, Mapa, ŽiviBoti)\\\hline
turn  & Z[4] alebo typy FORWARD, BACKWARD, LEFT, RIGTH & (InstruktionStack, Instructionpointer+1, Hitpoints, X, Y, NoveNatocenie)&(FrontaUdalosti, Mapa, ŽiviBoti)\\\hline
see & - &  (InstruktionStack, Instructionpointer+1, Hitpoints, X, Y, natocenie) & rovnaky efekt\\\hline %run nevediem! to iste ako preplanovanie za menej casu
see & [0-9]+ &(InstruktionStack, Instructionpointer+1, Hitpoints, X, Y, natocenie)& rovnaky efekt \\\hline 
shoot & [0-9]+, [0-9]+ &(InstruktionStack, Instructionpointer+1, Hitpoints, X, Y, natocenie)& (FrontaUdalosti.pop() insert( bot, strela ), Mapa, ŽiviBoti) \\\hline 
shoot & [0-9]+ &(InstruktionStack, Instructionpointer+1, Hitpoints, X, Y, natocenie)& (FrontaUdalosti.pop() insert( bot, strela ), Mapa, ŽiviBoti) \\\hline 
\hline
\newpage
JUMP & (InstruktionStack, Instructionpointer - LABEL, Hitpoints, X, Y, natocenie) & žiadny efekt až na preplánovanie \\ \hline               % inserting body of the table
CALL & FunctioName & (InstruktionStack + FunctionStack, Instructionpointer = AtFunctionStack, Hitpoints, X, Y, natocenie)& žiadny efekt až na preplánovanie \\\hline
EndFunction & (InstruktionStack - FunctionStack, Instructionpointer = AtPreviousState, Hitpoints, X, Y, natocenie)& žiadny efekt až na preplánovanie \\\hline
\hline                              %inserts single line
\end{tabular}
\end{table}

\subsection{Priebeh penalizacie za instrukciu}
\subsection{Detekcia zacyklenia}
\section{Vlastnosti robotov}
Výsledok programu, ktorého inštrukcie robot vykonáva, je závislý na konkrétnom type robota. Typ robota je definovaný niekoľkými parametrami, ktoré si bežný užívateľ samostatne na začiatku hry definuje == UholViditelnost, PolomerViditelnosti, Obrana, Hitpoints, TypStrely, VelkostPamate==\\
\begin{table}[ht]
\caption{Vlastnosti robota}   % title of Table
\centering                          % used for centering table
\begin{tabular}{ | c | p{10cm} |}            % centered columns (4 columns)
\hline\hline                        %inserts double horizontal lines
Vlastnost & Vplyv \\   % inserts table
%heading
\hline                              % inserts single horizontal line
polomer viditeľnosti & Robot dokáže na určitú vzdialenosť rozpoznať object, táto vlastnosť určuje, koľko políčok dopredu v priamom smere( v smere, akým je bot aktualne otočený) vidí. Tento počet sa pod iným uhlom o niečo zmení,\\\hline               % inserting body of the table
uhol viditeľnosti & maximálny rozptyl viditeľnosti - robot nevidí celý kruh okolo seba, ale iba určitú výseč \\\hline
Obrana & Obranne cislo bota. Pouziva sa pri strete s nejakou prekazkou. Viz kolizie. \\\hline
Hitpoints  & životnost bota, akonahle klesne na nulu, z fronty akcií sa jeho nasledujúca akcia odstráni a robot mizne do prepadlišťa dejin\\\hline [1ex]         % [1ex] adds vertical space
Typ strely & Strela je sa definuje podobne ako samotný bot, ale ma iné parametre\\\hline
\hline                              %inserts single line
\end{tabular}
\label{table:vlastnosti}          % is used to refer this table in the text
\end{table}

\begin{table}[ht]
\caption{Vlastnosti Strely}   % title of Table
\centering                          % used for centering table
\begin{tabular}{ | c | p{10cm} |}            % centered columns (4 columns)
\hline\hline                        %inserts double horizontal lines
Vlastnost & Vplyv \\   % inserts table
%heading
\hline                              % inserts single horizontal line
Odrážavost & definuje, ako moc sa strela od steny odrazí, v podstate je to to iste ako hmotnosť\\ \hline
Rýchlosť & definuje preplanovanie vo fronte akcií\\ \hline
Hitpoints  & dolet strely, ten sa znižuje s poctom tikov a súčasne kolíziami\\ \hline
Utok & Utocne cislo bota, pouziva sa pri utok, viz kolizie \\ \hline
\hline                              %inserts single line
\end{tabular}
\end{table}

Užívateľ si typ strely definuje tak, že dostane k dispozícií určitź počet bodv a tie musí medzi jednotlive volby rozdeliť\\
Bot premýšla tak, že sa nechá bežat dovtedy, kým z jeho program nenarazí na inštrukciu, ktorá by ovplyvnila svet, alebo kým neprsiahne timeout. Ak presiahne timeout, preruší sa, preplánuje vo fronte udalostí ďalej (presnejšie o TIMEOUT tikov ďalej)\\
Velmi špeciálny je prípad, keď užívateľ nenastaví robotovi žiadnu pamať. Takýto robot nie je schopný si zapatať jedinú premennú a ani vracat návratové hodnoty, teda v kóde programu by nemali byť definované žiadne funkcie, ale iba procedúry.Je nutné povedať, že funkcie a procedúry sa v žiadnom prípade nechovajú ako premenné a teda nezaberajú robotovi pamať. Teda v tomto prípade dojde k masívnemu využitiu rekurzií. Samotne vstavané príkazy sú však vlastne funkcie, ale kedže sa nikam nemajú priradiť, ich návratová hodnota sa zahodí. Nevýhodou tohoto typu bota je však skutočnost, že nemá povolené napriklad strieľanie, pretože to je funkcia, ktorá potrebuje za každých okolností parametre.
\section{Sietova komunikacia}
\chapter{Výber a popis testovaných algoritmov}
\section{Algoritmus 1}
\section{Algoritmus 2}
\section{Algoritmus 3}
\chapter{Testovanie}
\chapter{Vylepšenia}
\chapter{Záver}
\newpage
\addcontentsline{toc}{chapter}{\bfseries Literatúra}
\begin{thebibliography}{blablablabla}
\end{thebibliography}
\end{document}
