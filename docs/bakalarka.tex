\documentclass[12pt,notitlepage]{report}
\pagestyle{plain}
\frenchspacing 
\usepackage[utf8]{inputenc}
\usepackage[slovak]{babel}
\usepackage{a4wide}
\usepackage{amsmath}
\usepackage{amsfonts}
\usepackage{amssymb}
\usepackage{amsthm}
\usepackage{graphicx}
\usepackage{psfrag}
\usepackage{vmargin}

\begin{document}

\begin{titlepage}
\begin{center}
\vspace{1.5in}
{\rm Univerzita Karlova v Prahe\\
    Matematicko-fyzikálna fakulta}
\par
\vspace{0.7in}
{\huge \bf Bakalárska práca}
\par
\vspace{0.5in}
{Eva Pešková}
\par
\vspace{0.5in}
Codewars, vojna robotov
\par
\vfill
Katedra softwarového inžinierstva
\par
\vspace{0.5in}
Vedúci bakalárskej práce: Mgr. Tomáš Poch
\par
\vspace{0.5in}
Všeobecná informatika
\par
\vspace{0.5in}
2009
\end{center}
\end{titlepage}
\vfill
Prehlasujem, že som svoju bakalársku prácu napísala samostatne a výhradne s použitím citovaných prameňov.\\
Súhlasím so zapožičiavaním práce.\\
\par
V Prahe \today
\begin{flushright}
Eva Pešková
\end{flushright}

\newtheorem{definicia}{Značenie}

\tableofcontents
\chapter{Uvod}% na com sa to skusa, preco, co je hlanym  cielom, apod.
\input uvod.tex
\chapter{Analýza}
%\input sdl.tex
\input pravidla.tex
\section{Codewars a jeho alternativy}
\subsection {Robowars}
\subsection {Pogamut}
\section{Problem optimalnej strategie viacerych hracov}
\section{Prakticke vyuzitie}
Súčasná implementácia poskytule dosť širokú škálu použitia. Napríklad:\\
	\begin{itemize}
	\item V okamihu dobre vygenerovaného bludiska sa dá použit na vizualizáciu niektorých algoritmov, napríklad na vyhľadávanie v stromoch a tak zistiť optimálnosť algoritmu vzhľadom na architekturu stroja
	\item Program môže simulovať stratégiu robota pri vykonávaní určitéj úlohy, hodí sa povedzme na predmet typu EuroBot
	\item Pri istých parametroch môže simulovať pomerne zábavnú a chytľavú hru typu Herbert, ktorá bola náplňou súťaže ImagineCup v rokoch 2005-2008%todo zistit spravne roky 
	\end{itemize}
\input impl.tex
\input analyza.tex
\section {Burst tree uchovávajúci informácie o premenných a funkciách} 
\chapter{Sietova komunikacia}
\chapter{Výber a popis testovaných algoritmov}
\section{Algoritmus 1}
\section{Algoritmus 2}
\section{Algoritmus 3}
\chapter{Testovanie}
\input todo.tex
\chapter{Záver}
\newpage
\addcontentsline{toc}{chapter}{\bfseries Literatúra}
\begin{thebibliography}{99}
\bibitem{trees}
Justin Zobel,Steffen Heinz,Hugh E. Williams:\\
{\it Burst Tries: A Fast, Efficient Data Structure for String Keys} www.cs.mu.oz.au/~jz/fulltext/acmtois02.pdf
\bibitem{vm}
http://java.sun.com/docs/books/jvms/second\_edition/html/Overview.doc.html\#7565
\end{thebibliography}
\end{document}
