\documentclass[12pt,notitlepage]{report}
\pagestyle{plain}
\frenchspacing 
\usepackage[utf8]{inputenc}
\usepackage[slovak]{babel}
\usepackage{a4wide}
\usepackage{amsmath}
\usepackage{amsfonts}
\usepackage{amssymb}
\usepackage{amsthm}
\usepackage{graphicx}
\usepackage{psfrag}
\usepackage{vmargin}

\begin{document}

\begin{titlepage}
\begin{center}
\vspace{1.5in}
{\rm Univerzita Karlova v Prahe\\
    Matematicko-fyzikálna fakulta}
\par
\vspace{0.7in}
{\huge \bf Bakalárska práca}
\par
\vspace{0.5in}
{Eva Pešková}
\par
\vspace{0.5in}
Codewars, vojna robotov
7G\par
\vfill
Katedra softwarového inžinierstva
\par
\vspace{0.5in}
Vedúci bakalárskej práce: Mgr. Tomáš Poch
\par
\vspace{0.5in}
Všeobecná informatika
\par
\vspace{0.5in}
2009
\end{center}
\end{titlepage}
\vfill
Prehlasujem, že som svoju bakalársku prácu napísala samostatne a výhradne s použitím citovaných prameňov.\\
Súhlasím so zapožičiavaním práce.\\
\par
V Prahe \today
\begin{flushright}
Eva Pešková
\end{flushright}

\newtheorem{definicia}{Značenie}

\tableofcontents
\chapter{Uvod}% na com sa to skusa, preco, co je hlanym  cielom, apod.
\chapter{Analýza}
\input sdl.tex
\section{Codewars a jeho alternativy}
\subsection {Robowars}
\subsection {Pogamut}
\section{Problem optimalnej strategie viacerych hracov}
\section{Prakticke vyuzitie}
\input impl.tex
\section {Burst tree uchovávajúci informácie o premenných a funkciách} 
\chapter{Sietova komunikacia}
\chapter{Výber a popis testovaných algoritmov}
\section{Algoritmus 1}
\section{Algoritmus 2}
\section{Algoritmus 3}
\chapter{Testovanie}
\input todo.tex
\chapter{Záver}
\newpage
\addcontentsline{toc}{chapter}{\bfseries Literatúra}
\begin{thebibliography}{99}
\bibitem{trees}
Justin Zobel,Steffen Heinz,Hugh E. Williams:\\
{\it Burst Tries: A Fast, Efficient Data Structure for String Keys} www.cs.mu.oz.au/~jz/fulltext/acmtois02.pdf
\end{thebibliography}
\begin{tabular}{|l|c|c|c|c|c|} 
     \hline 
     {\bf Program} & {\bf Výsledek} & {\bf Odhad chyby} & {\bf Pøesnost} & {\bf Poèet uzlù} &  {\bf FLAG/IFLAG} \\ 
     \hline                    
     \hline 
     Maple  & \textbf{0,0890738559} & -         & -     & -  & -    \\ 
     \hline 
     QUANC8 & \textbf{0,0890738503} & 0,49e-08  & 1e-07 & 33 & 0,00 \\ 
     \hline 
     Q1DA   & \textbf{0,0890738520} & 0,56e-06  & 1e-05 & 60 & 0    \\ 
     \hline 
     \end{tabular}
\end{document}
