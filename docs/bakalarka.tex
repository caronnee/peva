\documentclass[12pt,notitlepage]{report}

\usepackage[utf8]{inputenc}
\usepackage{slovak}
\usepackage{indentfirst}
\usepackage{longtable}
%%%%%%%%%%%%%%%%%%%%%%%%%%%%%%%%%%%%%%%%%%%%%%%%%%%%%%%%%%%%%%%%%%%%%%
%% custom settings and definitions - begining

\def\CS{$\cal C\kern-.1667em\lower.5ex\hbox{$\cal S$}\kern-.075em $}

%% custom settings and definitions - end
%%%%%%%%%%%%%%%%%%%%%%%%%%%%%%%%%%%%%%%%%%%%%%%%%%%%%%%%%%%%%%%%%%%%%%

\pagestyle{plain}
\frenchspacing 
\usepackage[utf8]{inputenc}
\usepackage{a4wide}
\usepackage{amsmath}
\usepackage{amsfonts}
\usepackage{amssymb}
\usepackage{amsthm}
\usepackage{graphicx}
\usepackage{psfrag}
\usepackage{vmargin}
%\usepackage[left=4cm]{geometry} % nastavení dané velikosti okrajů

\begin{document}
\chapter{Uvod}
\section{motivacia}
\begin{itemize}
\item vyznam hier s programovatelnymi robotmi ( napad, vyuzitie )
\item ciel bakalarksej prace - vytvorenie hry, zameranie na boj
\end{itemize}
\section{ciele prace}
\begin{itemize}
\item vytvorenie sveta s vlastnostami, ktore by urcovali 'obtiaznost' hry
\item pre kazdeho hraca moznost rozne konfigurovatelnej postavy, co sa vlastnosti tyka
\item konfigurovatelna obtiaznost sveta vzhladom na napisany algoritmus
\end{itemize}
\chapter{Analyza}
\section{svet}
\subsection{Predstava sveta}
\subsection{dynamika sveta}
\section{Programovatelnost postavy}
\subsection{vplyv programovatelnosti na svet}
\subsection{prezentacia programovania}
\subsection{Vytvorenie noveho jazyka} (norma jazyka, co sa kde smie pouzivat)
\begin{itemize}
\item minimalna sila jazyka, co vsetko by mal minimalne zvladat
\item syntax jazyka
\item priklady pouzitia (napisane algoritmy)
\end{itemize}
\section {Interpretacia jazyka}
\subsection{mozne pristupy}
\begin{itemize}
\item cisty interprest ( mysleno nieco ako AWK, ciste text )
\item vlastny prekladac, ale do standartneho medzikodu (WTH?) alebo sdo strojoveho kodu (x86)(net MSIL, Java BYTECODE)
\item vlastnbe prekladac do vlastnmeho medzikodu
\item preklad do indeho, vyssie pouzitelneho jazyka  (??)
\end{itemize}
\subsection{struktura jazyka vzhladom na interpret}
\subsubsection{specifikacia instrukcii}
\subsubsection{preklad do robotstiny}
\subsubsection{Vysledna interpretacia}
\chapter{Implementacia}
\section{svet}
\section{preklad a intepretacia}
\chapter{porovnanie}
\chapter{zaver}
\section{Zhodnotenie splnitelnosti cielov}
\section{Mozne rozsirenie do buducnosti}
\tableofcontents
\addcontentsline{toc}{chapter}{\bfseries Literatúra}
\addcontentsline{toc}{chapter}{\bfseries Prilohy}
\begin{thebibliography}{99}
\bibitem{robocode} http://robocode.sourceforge.net/
\bibitem{mlaskal} http://ulita.ms.mff.cuni.cz/pub/predn/pp/
\bibitem{trees}Steffen Heinz and Justin Zobel and Hugh E. Williams,
    \emph{Burst Tries: A Fast, Efficient Data Structure for String Keys},
    ACM Transactions on Information Systems, 2002,
    volume 20, pp. 192--223.
\bibitem{vm} Tim Lindholm, Frank Yellin :\emph{The JavaTM Virtual Machine Specification, Second Edition}
\bibitem{simtool}http://mpherbert.codeplex.com/
\bibitem{quadtree}   Mark de Berg, Marc van Kreveld, Mark Overmars, and Otfried Schwarzkopf (2000). \emph{ Computational Geometry (2nd revised ed.) }. Springer-Verlag. ISBN 3-540-65620-0.  Chapter 14: Quadtrees: pp. 291–306.
\bibitem{usedPictures} Použitá galéria obrázkov: http://www.bghq.com/fft 
\end{thebibliography}
\appendix
\chapter{CD}
Prilohou bakalarskej prace je aj CD so zdrojovymi subormi.
\section{behove prostredie}
Pre beh aplikacie je nutne mat nainstalovane SDL verzie 1.2 alebo 1,3, (kniznu QT verzie 4), XML kniznicu libxml2 a kompiler gcc verzia 4
\section{Instalacia}
\begin{itemize}
\item Pre unix -> make
\item Pre windows -> prilozene solution
\end{itemize}
\section{struktura CD}
\begin{itemize}
\item sklada sa z tychto casti: language, image, vygenerovane subory (napisat az po upratani)
\item pre kazdy cim sa priblizne zaobera
\end{itemize}
\chapter{Zoznam pojmov}
\begin{description}
\item[objekt]
\item[robotstina]
\item[postava]
programovatelny robot, ktory sa objavi ako objekt vo svete
\item[robopamat]
	kvoli mylenie s pamatou fyzickou pamatou sa takto nazyva priestor, kde si postava uchovava hodnoty, ktore neskor pouzije
\item[instrukcia]
\end{description}
\end{document}
