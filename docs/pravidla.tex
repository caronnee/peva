\section{Úvod do Codewars}
V tejto sekcii by som chcela objasniť úlohu a pravidla simulácie Codewars, keďže sa na ne neskôr budem odkazovať. \\ 
Ako názov naznačuje, v Codewars sa proti sebe postaví niekoľko užívateľom napísaných programov. Tieto programy predstavujú logiku robota a sú opakovane vykonávané v prípade, že boli vykonané všetky inštrukcie, ktoré program obsahoval. Každý hráč(užívateľ) si po spustení programu vytvorí minimálne jedného virtuálneho robota na hranie, dalších si neskôr môže pridať. Pre jednoduchosť predpokladajme, že každý užívateľ vlastní práve jedného robota a že máme viacero hráčov. Úlohou hráčov je napísať pre každého svojho robota vlastnú logiku, ktorou sa bude riadiť. Po napísaní tejto logiky v špeciálnom jazyku stvoreným pre Codewars sú títo roboti vypustení do dopredu pripraveného prostredia. \\
\indent Okrem robotov sa v Codewars vyskytujú aj ďalšie objekty, na ktoré má hráč pomerne malý vplyv. V súčasnosti sú to strely a steny. Vplyv robota na tieto objekty bude vysvetlený neskôr pri samotnom popisovaní vlastností a priebehu hry. \\%TODO  referencia na tu kapitolu
\indent Robot je vlastne iba postava, ktorou hráč môže pred samotným bojom pripraviť. Príprava spočíva okrem vybavenia ho vnútornou logikou aj v úprave vlastností robota a jeho zbrane. Tieto vlastnosti sa rozdeľujú do určitých tematicky podobných skupín (sekcií), v rámci ktorých sa môžu vlastnosti meniť. Sú to tieto:
\begin{itemize}
\item Vlastnosti pojednávajúce o stave robota
\begin{itemize} 
\item Dĺžka života : je číslo označujúce životaschopnosť robota. Štandartne sa znižuje v okamihu, keď na miesto, kde stojí robot, chce pristúpiť iný objekt. Existuje možnosť, že sa v tomto prípade toto číslo aj tak nezmení, viz \ref{kolizie} 
\item Polomer viditeľnosti: určuje vzdialenosť, do akej maximálne robot môže robot rozoznávať ostatné objekty
\item Uhol viditeľnosti: určuje kruhovú výseč, vnútri ktorej robot rozoznáva objekty.
\item Veľkosť pamäte : určuje maximálny počet premenných, funkcií a návratových hodnôt, ktoré si robot pamätá a tým obmedzuje napríklad hĺbku rekurzí, keďže každá funkcia má nejaký návratový typ, ktorý si robot musí pamätať, vrátane typu 'Void', teda prázdno
\item Obranné číslo: určuje odolnosť robota vočí iným objektom
\item Útočné číslo: určuje útočnosť robotovho tela pri kolízií s inými objektami
\end{itemize}
\item Vlastnosti týkajúce sa akcií na diaľku
\begin{itemize}
\item Ďalekonosnosť zbrane: určuje, ako daleko je možné náboj vystreliť. Ak zbraň prejde viac ako daný počet políčok, automaticky zaniká. Dá sa tak zameniť so životnosťou zbrane
\item Útočnosť zbrane: určuje ako moc je zbraň schopná ublížiť objektom, viz \ref{kolizie}
\end{itemize}
\end{itemize}
Toto rozdelenie znamená, že hráč môže napríklad zvýšit robotovu odolnosť voči stretu s inúmi objektami, ale súčasne mu potom zostáva menej na zvýšenia životaschopnosti robota. Počet bodov, ktoré sa v rámci sekcie prerozdeľujú, je vždy v každom oddieli pre každého hráča rovnaký (T.j. každý hráč prerozdelí X bodov v rámci prvej sekci a  Y v rámci druhej, kde X, Y nemusia byť nutne navzájom rôzne čísla). Rozdelenie vlastností do sekcií nemá väčší než estetický význam. Predstava, ze na úkor očí bude puška alebo podobný nástroj lepšie strielať, nie je príliš intuitívna a užívateľa by to mohlo pomerne zmiasť. Ďalej je toto rozdelenie výhodné aj z toho dôvodu, že užívateľa núti znížiť obranyschopnosť robota. Ako vyplýva z popisu kolízií, ak by bola obranyschopnosť robota príliš vysoká, nič by honemalo šancu zasiahnuť. Potom výsledok celej simulácie by bol, ze buď vyhrá hráč s vyššiou obranyschopnosťou, alebo hra skončí remízou, keďže ani jeden z hráčov nemá dostatočné prostriedky zraniť druhého hráča. \\
Čo sa týka samotných čísel X a Y, je možné si ich samostatne definovať, a to práve kvôli veľkostiam pamäte. Tieto čísla sú potom záväzné pre každého robota.
\\Hlavnou úlohou hráča je však napísať algoritmus, podľa ktorého sa bude robot správať. Optimálny algoritmus spraví to, robot ako jediný prežije, t.j väčšinu svojich nepriateľov zničí alebo ich prinúti zničíť sa(napríklad navedením na vlastnú strelu alebo podobne).

\subsection{Vlastnosti stien}\label{walls} % vseobecne oba rozne vlastnosti, ake maju steny, k comu to je
Steny sú objekty sveta, cez ktoré robot nevidí a ktoré môžu napríklad vychýliť dráhu strely. Nie sú však len statické, exituje viacero druhov stien, ktoré budú komunikovať s mapou. Sú to:\\
\begin{itemize}
\item Zvláštne políčko Start, ktoré sa používa iba pri generovaní mapy, označuje políčko, kde majú roboti na začiatku simulácie stáť. V prípade, že to políčko sa na mape nevyskytuje, potom sú roboti rozmiestní náhodne.
\item MovableWall, touto stenou je možné pohnúť, ak na políčku za za ňou v smere pohybu nič nie je.
\item TrapWall. Táto stena vždy na náhodný čas zmizne a znova sa objaví. V prípade, že sa v okamžiku jej objavenia je na jej políčku nachádza iný objekt, program sa chová, akoby nastala kolízia, viz \ref{kolizie}.
\item SolidWall, obyčajná stena bez špeciálnych vlastností.
\end {itemize}
Steny sa dajú kombinovať, to znamená, že napríklad stena sa môže pohybovať alebo miznúť alebo súčasne. Pri generovaní mapy sa toto dosiahne položením steny na polííčko, ktoré už obsahuje iný typ steny.

\subsection{Kolízie}\label{kolizie}
Kolízia je stav, keď na jedno miesto sa chcú dostať dva objekty. Miesto v našom príprade je jedno políčko mapy. V prípade kolízie jeden alebo druhý objekt utrpí a to buď útočník, ktorého útočnosť bola menšia ako obranyschopnosť brániaceho sa objektu(teda neprenikol obranou, ako výsledok sa mu zmení smer, ktorým sa doteraz pohyboval), alebo utrpí brániaci sa objekt, ak jeho obrana nebola dostatočná. V tom prípade dostane tento objekt odpovedajúce zranenie. Toto zranenie sa vypočíta ako rozdiel útočnosti útočníka a obranyschopnosti brániaceho sa objektu. To, ako sa po úspešnom útoku zachová útočník, závisí na jeho type. Ak je to napríklad strela, zanikne alebo sa odrazí od steny, ak je to stena, svojím pohybom útočí ďalej, ak je to hráč, závisí na jeho algoritme.
\subsubsection{Detekovanie kolízií}%TODO referencia
Detekovanie kolízií je jedno z najkritickejších miest pri vytváraní grafickej hry. Nejde pri tom ani tak o samotné spracovanie výsledkov kolízie ako uloženie samotných objektov do štruktúry, ktorá pomerne rýchlo vyhľadá objekty možných kolízií. Existije na to niekoľko metód.\\
Najjednoduchšia metóda, ktorá sa dá použiť, je samozrejme uložiť si do poľa všetky objekty a systematicky prechádzať každú dvojicu, či náhodou nekoliduje. Kód vyzerá pomerne jednoducho:\\ $ for(std::vector<Object>::iterator = objects.begin();\\ iterator!=objects.end();\\ iterator++)\\ \\ checkForCollistion(iterator, object);$\\
Toto riešenie je ale v porovnaní s ďalšími navrhovanými algoritmami dosť pomalá. Pre málo objektov je však pomerne výhodná.\\
Ďalším navrhovaným riešením je rozdelenie priestoru do mriežky. Každá buňka v tejto mriežke môže obsahovať niekoľko objektov, ktoré treba testovať na kolíziu. V tomto okamihu sme podstatne zmenšili počet objektov, ktoré budeme testovať a teda môžene použiť prvý spôsob na všetky objektu, ktoré sa nachádzajú v rovnakej buňke. Keďže v okamihu pohybu môže objekt prejst z jednej buňky do druhej, je tiež potrebné testovať na zrážku s objektami vo vedľajších buňkách. Veľkosť plochy, aký pokrývajú jednotlivé buňky, je dobré zvoliť v závislosti na rýchlosti objektov a počte objektov. Ak by bola rýchlosť objektov príliš vysoká, mohlo by sa stať, že objekt neprejde len do susednej bňky, ale ešte do buňky za ňou, takže by sa museli porovnávať okolité buňku po ceste, čím by mohla výkonnosť klesnúť, nie však za predpokladu, ze že je v každej buňke málo objektov. Táto metída bola pre jednoduchosť implementovaná.\\%TODO obrazok
Dosť často je využívaným spôsobom pre riešenie kolízií v 2D (3D) hrách je štruktúra zvaná quadtree. Quadtree je strom rozdeľujúci priestor na niekoľko častí podobne ako v mriežkovom spôsobe, na základe dostatočnej 'blizkosti' objektu, rozdiel spočíva v prístupe k informáciam a dynamickým vytváraním ďalších buniek. Na začiatku tvorí celý priestor jednu veľkú buňku. V okamihu, keď sa do buňku dostane istý počet objektov považovaný za maximálny, buňka sa rozdelí. Podobne, pokiaľ objekt prejde z jednej buňku do druhej a v pôvodnej už nezostáva nič, buňka zanikne (zostane už iba naradená, väčšia buňka). Samotná kolízia sa preráta podobne ako v predchádzajúcej metóde. Pekný tutoriál k implementácií Quadtree sa dá  nájst v \cite{quadtree}
\subsection {Priebeh hry}
Hra sa začína vhodením robotov do prostredia v poradí, v akom boli pridaní a roboti by mali začať vykonávať svoj program súčasne. To implementačne znamená, že buď každý nový robot dostane vlastné vlákno, alebo sa určí spôsob zoradenia robotov tak, aby nesimultánnosť bola vidieť čo najmenej. Spôsob preplánovania jednotlivého súčastí CodeWars je popísaný neskôr, nateraz stačí povedať, že každý jednotlivý kus kódu programu robota zaberá istý čas (napríklad na premiestnenie sa alebo vypočítanie premennej), a teda sa robot môže dostať opať na radu(vykonať nasledujúcu inštrukciu) až v okamihu, keď tento čas pominie. Popis inštrukcií a čas, ktorý zaverajú, je popísaný v \ref{penalizacia}. Robot teda môže sa pohnúť a niekoľko políčok, vystreliť pod uhlom alebo na urcté miesto, otočit sa, rátať premenné, používať podmienené príkazy a skoky typu while, if, break, continue. Hra končí v okamihu, keď aspoň jeden z robotov dosiahne 'víťazného stavu'. Defaultne je to zabitie každého protivníka, je ale možné priamo v kóde definovat niekoľko variánt.\\
Definovanie vítazného stavu sa v pro písaní programu pre robota objaví ešte pred funkciami v tvare:
\begin{itemize}
\item $Killed\ [>, < <=, >=, !=, ==]\ N, N \in Z$ , kde Killed je pocet zabitých nepratieľov a do tohoto počtu sa nezapočítavajú rozbité steny ani strely
\item $Killed( N ), N \in Z$ , kde N je identifikačné čislo robota. Toto číslo je každému robotovi pridelená automaticky, ale pre lepšiuidentifikáciu, napríklad pri sieťovom pripojení, je možné i roboto toto číslo explicitne definovať napísaním $ ID\ n, n\in N$ na rovnak=ych miesach ako tieto pravidlá
\item $Visit([X,Y], [Z,W],..)$, kde X,Y sú súradnice miesta, ktoré musí robot prekročiť, v akomkoľvek poradi
\item $VisitSequence([X,Y], [Z,W],..)$, kde X,Y sú súradnice miesta, ktoré musí robot prekročiť, práve v tomto poradi. V prípade, že je definovaných viac takýchto inštrukcií, spracovávajú sa všetky simultánne, T.J ak má pravidlá:\\
$VisitSequence([10,10],[20,20])\\VisitSequence([20,20],[20,30])$ a pristúpi na políčko [20,20], splní sa v druhej sekvencii prvé miesto. Ak robot už predtým navštívil políčko [10,10], potom je kompletne splnená prvá podmienka a časť druhej
\end{itemize}
V týchto podmienkach je ešte možné použit premennú 'Start', ku ktorej sa pristupuje ako k poľu a označuje štartovnú pozíciu i-teho robota.\\
Kód, ktorý by simuloval populárnu hru 'Capture the flag' by napríklad vyzeral takto pre robota s ID 0:\\
$VisitSequence(start[1],start[0]) \\ main()\\\{\\//kod robota\\\}\\$
