\chapter{Porovnanie}
Existuje mnoho nástrojov, ktoré sa venujú súťažne algoritmom zameraným na programovanie robotov. Z typických zástupcov môžeme uviesť uz spomínanych RoboCode, C++ robots, ako aj MindRover, Grobots a množstvo ďalších v \cite{koth}. Väčšinou sa zameriavajú na prežitie v aréne. Podla corewar terminológie sa tento cieľ nazýva "King Of The Hill" ( ďalej len KotH )
\\
% citacie by mali byt uvedene v Literature chysta sa
%http://sumost.ca/steve/games/    
%http://www.gamerz.net/c++robots/
%http://grobots.sourceforge.net/
%a teraz nasleduje porovnanie
Stručne porovnáme výsledný program bakalárskej práce s uvedenými aplikáciami, zhrnieme výhody a nevýhody ich prístupov a čo nové prinásame. Obmedzíme sa len na niekoľko charakteristických nástrojov ponímania sveta podobných CODEWARS. Tie budú demonštrovať, aký koncept  CODEWARS použila a ako ho zmenila/vylepšila.\\

C++ robots\cite{crobots} je veľmi podobná hra o prežitie. Roboti sú písaní v C++ a majú k dispozícií jedinú zbraň - kanón. Svet je ale veľmi jednoduchý, je to priestor 100x100m ohraničený stenami.  Celá hra je na rozdiel od CODEWARS koncipovaná ako turnaj, robot si postupne vyberá súperov. %poment, scheknem, ci nekecam
Codewars  naviac ponúka možnosť pustiť všetky algoritmy naraz s tým, že hráč môže využívať aj ostatných  robotov, aby za neho spravili špinavú prácu. \\%a viac by som nepisala..== viac sa mne neche

ARES\cite{ares} je typická obranno-útočná hra, ktorá je ale hlavne určená programátorom. Svet sa mení veľmi dynamicky a roboti nevedia vôbec nič o svete. Len  sa predstavujú výsledky útoku a podľa toho reagujú. Codewars je z tohoto pohľadu úplne opačná aplikácia, roboti vedia veľmi dobre mapovať svet a spoznajú nepriateľa. Spoločný prvok tak môžu mať len v spôsobe interpretácie algoritmu po kolách. V Codewars bol tento koncept dovedený ad absurdum zavedením plánovača.\\
RoboCode je tiež zložitá hra, ktorá sa dlho vyvíjala. Je určená pre užívateľov, ktori začinajú s programovacím jazykom Java. Svet sa riadi svojími zvláštnými pravidlami pre streľbu a víťazenie. Víťazný algoritmus je taký, ktorý získa najviac bodov. Codewars bolo silne inšpirované týmto programom.  Líšia sa ale významne v spôsobe, akým vykonávajú algoritmus ( RoboCode sa snaží o paralelizáciu ) a definovaním cieľa. Codewars oproti Robocode prináša možnosť prispôsobiť si robota pomocou vlastnosti jeho algoritmu a s možnostou napisat aj čisto mierumilovného, no chytrého robota.\\ %vyborne!:P

Koncepty prijaté v jednotlivých hrách sú popísané v tabuľke \ref{table::porovnanie1} a v \ref{table::porovnanie2}. Tabuľka obsahuje iba tie hry, ktoré sa od seba vyrazne líšia. 
\begin{table}
\centering
\caption { Charakteristiky porovnavanych hier } %to bude musiet rozdelit asi na dve tabulky...
\begin{tabular}{|c|c|c|c|c|}
	&Svet	&zbrane& ciele & obmedzenia \\
	&	&	& 	&na algoritmus 		\\
\hline \\
POGAMUT & 3D & strely & rôzne & žiadne \\
ARES & 1D & zdieľaná pamäť & KotH & veľkost pamäte \\
Codewars &2D & strely, roboti & voliteľné & voliteľné \\
C++Robots &2D& strely, roboti & KotH & žiadne  \\ 
\end{tabular}
\label{table::porovnanie1}
\end{table}

\begin{table}
\centering
\caption { Charakteristiky porovnavanych hier } %to bude musiet rozdelit asi na dve tabulky...
\begin{tabular}{|c|c|c|c|}
	&pohyb &objekty&jazyk \\
\hline \\
POGAMUT & komplexný &množstvo&Java \\
ARES & vykonanie inštrukcií & data & assembler\\
Codewars & jednoduchý 2D& variácie stien &jednoduchý \\
C++Robots & jednoduchý & & C++ \\ 
\end{tabular}
\label{table::porovnanie2}
\end{table}

\chapter {Záver}
\subsection{Zhodnotenie splnenia cieľov}
V našom programe sa podarilo implementovať dvojrozmerné prostredie pre robotov, kde platia základné fyzikálne. Svet je navrhnutý tak, aby jeho zmena spôsobila rozdielne nároky na algoritmy a tým zvyšovala obtiažnosť hry v zmysle nároku na úspešnosť algoritmu. Toto je docielené pomocou plánovačov. \\
Program taktiež priniesol nový pohľad na ciele, ktoré sa dajú v dvojrozmernom svete dosiahnuť. Užívateľovi je dovolené meniť vlastnosti svojho robota v závislosti na type algoritmu.\\
Program je pripravený na testovanie algoritmov. Z časových dovodov sa však nepodarilo otestovať algoritmy, ktoré su považovanie v niektorých hrách za dobré v danom rozsahu. 

\subsection{Možne rozširenie CODEWARS}
V tejto sekcii zhrnieme rozširenia, o ktorých už bola zmienka v texte, ich prínos a možný smer rozvoja problematiky. \\
\subsubsection{Rozšírenie vzhľadom na jazyk}
Jedným zo spomenutých rozšírení je navrhnutie viditeľnosti tak,  aby robot mal asymetricke 'oči', (t.j priamka definovaná smerom robota), v ktorom je práve otočený a nebude rozdelovať výseč na dve rovnaké časti. Tento škuľavý robot bude do nejakej strany vidieť viac ako do druhej. Dokonca je možné pripustiť extrém, kedy by robot videl iba za seba a tým by miatol súperov. Chôdza do akejkoľvek strany je povolená a tak jediný výsledok by bol, že by nesedelo zobrazovanie. Robot nemôže spoľahnúut ani na to, ako je otočeny. Algoritmus by musel veľmi sofistikove testovať, kam robot vidi, alebo si prinajmnšom často a správne tipnút. Takto hendikepovany robot je skutočnou výzvou najmä pri hre skúseného hráča so začiatočníkom. Preto toto rozšírenie je  hodné pozornosti. \\ %myslis?mne to pride ako zrada na hracovi:P <<<<TAKY byva aj realny zivot !
Ďalšou možnosťou je nechať užívateľa definovať, aký plánovač bude konkrétny robot použivať. Možno očakávať, že výsledky by mohli byť zaujímavé, nakoľko to vedie k naprogramovaniu robota, ktorý má podla všetkého rýchlejší algoritmus.\\
\subsubsection{Rozširenie vzhľadom na jazyk}
Jazyk robota aktuálne nepodporuje deklarovanie premenných, ktore boli vo funkcii už niekedy deklarované. Toto chovanie je síce pochopiteľné, avšak programátorsky nepríjemné. Preto by sa dalo uvažovať o primeranej náprave. %PEKNE:D \\
Jazyk dostatočne pokrýva základné požiadavky na popísanie chovania robota. Avšak je príliš nízkoúrovňovýy, uživateľ si musi mnohé detaily ošetrťt sám. To na jednej strane môže pôsobiť blahodárne na vymyšľanie stratégie, na druhej strane môže uživateľa znechutiť napr. to, že si jeho robot zabúda všimať, že je ostreľovaný. Preto by bolo možné rozširť jazyk o funkcie, ktoré sa automaticky spustia pri vyvolaní udalosti. Toto by však vyžadovalo  hlbší zasah do kódu, pretože robot ako objekt a jazyk ako hýbateľ robota sú  implementované ako dve nezávislé entity. Bolo by nutne implementovať príslušné komunikačne rozhranie. \\%otazka, dalo by sa to u mna jednoducho? Podla mna NIE
Ďalšou alternatívou je pri implementovani reakcií na udalosti nechať uživateľa vopred definovať tieto udalosti, na ktoré bude robot reagovat (napriklad "on seeEnemy()>0"). Pri dobrom komunikačnom protokole by stačilo súcasne s vykonavanim kódu kontrolovať zoznam podmienok. Obmedzenie počtu udalosíi, na ktoré robot môže reagovať, by tiež mohlo prispieť k zaujímavým úvahám o strategiách 

