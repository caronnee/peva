\chapter{Úvod}
%chcem napisat, ze sa budem venovat algoritmom chovania robotov
%nieco ako uvod do problematikly, karela herbert
Bakalárska práca sa zaoberá, ako názov naznačuje, súťažením algoritmov. Súťaženie však musí prebiehať zábavnou a zrozumiteľnou formou.
%( tot prec-to aby to nebola jednorazovka, kchceme aby sutazili, ze? zrozumitelne definovanie problemov, ktorym sa budeme venovat, vysledok, aky to ma predstavovat ).
Vhodným suťažiacim kritériom je napísanie algoritmu chovania robota, ktorý  bojuje s inými robotami o život. \\ % BLE
%Co asi chcem nakodit a preco - ne, je to moc skoro
%NEJAKE OBECNE KECY O ALGORITMOCH CHOVANIA, HERBERT, KAREL, chcem napisat, ze k nieomu to, ze si moze naprogramovat vlastneho robota, vazne bude
%Karel vzbudl zaujem o programovanie
\indent Jedným z prvých programov, ktorý používal koncept programovateľného robota, je KAREL\cite{Karel}. Tento program bol vytvorený pre podporu výučby programovacích jazykov. Príležitosť potrápiť svoje schopnosti zdanlivo banálnymi príkladmi jednoduchým jazykom (napr. napíšte program, po skončení ktorého bude robot v strede svojho sveta), viedla k vytvoreniu súťaží. Doteraz je populárna hra C++ robot\cite{crobots}, prebiehajúja pomocou mailovej konferencie. Tie rôznymi obmedzeniami, kladenými na algoritmus nútili programátora pristupovať k problému kreatívne.\\
%herbert ako aplikacia algoritmu
\indent Dobrým príkladom takejto súťaže je hra HERBERT\cite{herbert}, kde bolo cieľom prejsť v neprerušenej postupnosti všetky biele políčka v šachovnicovom svete. Zistiť postupnosť príkazov, ktoré sú riešením, bolo triviálne. Problémom však bolo zapísať ich pomocou rekurzií a s čo najmenšim počtom použitých príkazov.
%tu chcem smerovat k tomu, ze chceme, aby sme boli schopni napisat algoritmus a obmedzit svet.
\section{Motivácia}
Ako sme naznačili v úvode, vymýšľanie stratégií chovania robota a následné pozorovanie výsledku je dobrým spôsobom, ako si overiť svoje schopnosti zábavnou a hravou formou. V predstavených hrách je ale jediným súperom zadaný problém. Súťažiaci tak rátajú s pevne danými dátami ( rozostavenie políčiek a pod. ). Na podobnom princípe je založené aj súbežne pustenie algoritmov s tým, že ich vykonávanie môže ostatným škodiť. Algoritmus súťažiaceho musí rátať s nerovnakým prostredím vo virtuálnom sveta a môže dokonca profitovať zo znalosti stratégie súperov. Preto sa v tejto bakalárskej práci zaoberáme algoritmami robotov, ktoré sa budú vykonávat paralelne v rovnakom prostredí.\\
%robocode, ako vyzera, a ze je optimalnym
Smerom, ktorým by sme sa chceli uberať, je hra ROBOCODE\cite{robocode}. Užívateľ programuje v jazyku Java tanky. Tie hľadajú a ničia nepriateľské tanky.  Vo virtuálnom svete sú nastavené základné obmedzenia, s ktorymi musí uživateľ počítať. Napr. čo sa stane, ak strieľa tank príliš často a pod. Predstavme si extrémy, ktoré existujú v takýchto hrách a definujme smer, v ktorom sa bude uberať bakalárska práca.
\begin{itemize}
\item ARES - hra dvoch hráčov. Odohráva sa na mieste simulujúcom pamäť počitača. Úlohou hráča je naprogramovať robota v tomto prostredí v strojovom kóde (assembleri). Kritériom úspešnosti je program, ktorý sa bude vykonávať ( prakticky hocijaký ), a hodnotiacou funkciou je čas, za ktorý program pobeží. Program skončí v okamihu, keď sa pokúsi vykonať neplatnú inštrukciu (napr. delenie nulou, skok na adresu nula, prázdnu inštrukciu ).  Cieľom je prepisať pamäť takým spôsobom, aby sa druhý program ukončil. Hráč dopredu nepozná ani program protihráča, ani dáta, ktorými je inicializovaná pamäť. Hráči sa však môžu dopredu dohodnúť. Oba programy sú podobne ako v reálnom svete uložené v pamäti počitača a keďže pamät je zdieľaná, môžu si navzájom prepisovať dáta alebo dokonca inštrukcie. Extrém, ktorý tento koncept prináša, je: %sumarizacia
\begin{itemize}
\item použitie jednorozmerného priestoru ( pole pamäte ) 
\item obmedzenie na počet hráčov ( 2 )
\item nutnosť poznať do hĺbky assembler, jazyk, v ktorom je zapísaný algoritmus
\item možnosť zásahu do algoritmu ostatných hráčov a teda jeho zmena
\item objekty vyskytujúce sa v hre sú iba dáta a inštrukcie
\item hra sa odohráva na najnižšej možnej úrovni -  nie sú tu telá robotov, ale iba priamo program.
\end{itemize}
\item POGAMUT je už vysoko komplexná hra niekoľkých robotov. Algoritmy sa dajú programovať v Jave, ktorý umožňuje používať špeciálne znaky jazyka, ako je napríklad preťaženie, dedičnosť, atd. Prostredie je trojrozmerné a tým majú roboti možnosť širokej škály pohybov, skákania po stene, pohľady hore a dole, strely do boku atď. Spôsob, akým sa ubližuje robotom, je intuitívny - robot vystreľuje obmedzené množstvo striel a má na výber viac zbraní, ktoré sa líšia presnosťou zásahu. Hráč má dokoca možnosť ručne riadiť vlastného robota proti naprogramovanému a tým otestovat vhodnosť jeho algoritmu. \\
Extrémom v tomto kontexte sú:
\begin{itemize}
\item trojrozmerný priestor, ktorý ponúka množsto možností, ako realizovať pohyb - lezenie po stenách, skákanie, gravitácia
\item množstvo objektov pôsobiacich na robota, napr. úkryty, strely, vyhýbanie sa strelám, obehnutie prekážky, hľadanie vhodných ciest
\item zobrazovanie je dobre zrozumiteľné pre pozorovateľa, používajú sa základné fyzikálne javy 
\item možnosť obmedzeného videnia robotov, robot sa môže schovať alebo byť tak ďaleko, že ho iný robot nezaregistruje
\end{itemize}
\end{itemize}
Uvedené hry uspokojujúco spĺňajú základnú problematiku boja algoritmov. V ARES-e je kritériom zostať nažive podobne ako v ROBOCODE, v POGAMUT-e je naviac vopred dané kritické množstvo protivníkov. Hodnotiaca funkcia je rýchlosť, kto skôr splní cieľ, vyhráva.\\
%Zostrojenie takejto hry ale vyžaduje podrobnejši prehľad o nárokoch na jazyk, svet a samotných robotov. Preto sa zameriame na nasledovné charakteristiky týchto hier:
%\begin{description}
%\item[Priestor hier] Kým v ARES-e ide o 1D priestor (jedna veľka pamät - pole), boj vPOGAMUT-e sa odohráva v 3D priestore. S tým súvisí pohyb po ireálnom svete. 3D priestor má omnoho viac možnosti, ako realizovat pohyb. Je nutné zváziť, či bude povolené lietanie, padanie, pohľad zhora, zdola, vrhanie zbraní zboku, a v akých smeroch sa objekty sveta odrážajú a podobne. V ARES-e sa o pohybe, ako ho pozname ( plynulý prechod z miesta A na miesto B ) nedá ani hovoriť, pretože všetky akcie súviace so svetom sú inštrukcie a zmeny v pamati. 
%\item[Kritéria úspešnosti] V ARES-e je jasné, že hra skonči, keď háač nedokáže nadalej vykonávať svoj program . V POGAMUT-e je situácia o poznanie horšia: Pri programovaníi robota sleduje programátor (hráč  dva ciele:
%(a) buď naprogramovať takého robota, ktorého zdolať bude výzva, alebo 
%(b) naopak takého robota, o ktorom sa všeobecne vie, že síce bude poraziteľný, ale nie je ľahké ho obísť. To znamená naprogramovať takého robota, ktorého zdolávať bude výzvou, ale ktorý bude mať chyby, ktorých sa da využiť.
%\item[Spôsoby boja]
%POGAMUT na rozdiel naviac od ARES-a implementuje omnoho viac spôsobov, ako ublížiť robotovi, od rôznych zbraní po odrazenie guliek, rýchleho spadnutia na zem, atd. Jediným spôsobom, ako je možné v ARES-e poškodiť protivníkovi, je. prepísať mu tú časť pamäte, o ktorej je dôvod predpokladať, že ju bude v dohľadnej dobe potrebovať. Vyhodnotenie algoritmu nie je tak možné po častiach, ale až po skončeni celej simulácie. V POGAMUT-e je možné rozlíšiť  už v priebehu simulácie, ako a či robot zasiahol protivníka.
%\item [Vykonávanie programu] Ďaľšim prístupom, ktorý je v vytváraní hry dôležitý, je aj spôsob, v akom poradí sú akcie hráčov interpretované. Aby ostatní hráči neboli znevýhodnení, je vhodné vykonávať jednotlivé časti nezávisle od na ostatných robotov podľa jednotných pravidiel pre všetkých. V ARES-e je to jednoduché, hra prebieha po kolách. Každé kolo znamená vykonanie aktuálnej inštrukcie, čo je to spravodlivé pre všetkých hráov. V POGAMUT-e je nutne zaistiť paralelizáaciu, aby robot nebol závislý na vykonávani programu ostaných robotov.
%\end{description}
\section{Ciele práce}% (ilúzie - tvrdá realita)}
%tu to zmenit na ciel, na ktorom sme sa vraj dohodli - super
{\bf Cieľom bakalárskej práce je umožniť užívateľovi naprogramovať robota, ktorého algoritmus chovania bude súťažiť s ostatnými robotmi v prostredí, kde si navzájom môžu ubližovať. V tomto prostredí sa budú dať definovať aj rôzne nároky na algoritmus. Tým je myslené napríklad obmedzenie použítia niektorých prvkov, nároky na spotrebu zdrojov a pod. }\\ %ciele neskôor
% a potom sa na to odkazovat. Snad je to to, co sakra chcel.
\indent Výsledný program bakalárskej práce je nástroj, v ktorom môže užívateľ meniť virtuálny svet, kde sa odohrávaju súboje, naprogramovať robota a zistiť úspešnosť napísaného algoritmu. Program by mal byť napísaný tak, aby bol portabilný.\\ 
Predstavili sme si dve hry, ktoré istým spôsobom predstavujú extrémy v tomto obore. Práca by mala spojiť niektoré koncepty z týchto extrémov s tým vytvoriť nové súťažné prostredie. %Moc sa mi nelubi...
\newline
Preto je potrebné sa zamerať najmä na: %upresnenie cieľov
\begin{description}
\item [Vytvorenie virtuálneho sveta]
Naprogramovaný robot bude žiť vo virtuálnom svete, v ktorom sa odrážajú fyzikálne zákony a kde je prirodzené sledovať postup algoritmu. Súčasťou sveta budú objekty, ktoré interagujú s robotmi a tak prinášajú do tvorby stratégií komplikovanejšie prvky. V ARES-e reprezentujú tieto objekty dáta uložené vo virtuálnom svete (pamäti). V prípade POGAMUT-u sú to steny, teleporty, priepasti, strely a pod. Codewars sa pokúsi spojiť obidva tieto koncepty, to znamená, že na hráča bude kladený nárok aj z h)ladiska prostredia aj z hľadiska algoritmu.
%Definovanie týchto objektov zahŕňa aj ich vlastnosti, napr. priehľadnosť, existencia predmetov na dobijanie zdravia, streliva a pod. 
Predpokladá sa, že takto vytvorený virtuálny svet bude možné upravovať a vytvárať, aby algoritmy nemuseli byť napisané "na telo" jednej mapy/počiatočnému stavu sveta. Hráč bude mať možnosť ovplyvniť/zmeniť správanie tohoto sveta.
\item [Dynamika virtuálneho sveta] 
Naprogramovaní roboti budú mať možnosť bojovať,  t.j. si ubližovať, výsledok útoku bude známy v okamihu ublíženia. Tým sa ľahšie vyhodnotí program. To je smer, v ktorom sa budeme približovať hre POGAMUT, keďže v ARES-e hráč nevie, kedy a ako ublížil robotovi.
Roboti vo svete sa budú pohybovať všetkými smermi a interagovať s ostatnými objektami vo svete. Pre ľahšiu zrozumiteľnosť vývoja virtuálneho sveta sa objekty budú riadiť základnými fyzikálnymi javmi, ako je napríklad odraz. Hráč by mal mať tiež voľbu útoku na blízko aj na diaľku kvôli lepším strategickým možnostiam. % Inak by sa hra zvrtla na t "najdi robota a dufaj, že ma slabý útok".
\item [Životný cyklus robotov]
Život robotov bude začínať vstupom do virtuálneho sveta a končiť jeho opustením. Možnosť nejakého znovuzrodenia ako je to v hre POGAMUT sa nebude pripúšťať, ale môže byť otázkou ďaľšieho rozšírenia. Víťazný robot ostáva živý. Životný cyklus robota sa bude dať naprogramovať pomocou nejakého programovacieho jazyka, ktorý bude dostatočne zrozumiteľný aj pre laika. %Tymto prístupom kombinujeme prístup Pogamutu a Aresa v ich náhľade na svet
\item [Vlastnosti robotov] 
Ani jeden z uvedených programov ale nemá možnosť špecifikovať, aké budú jednotlivé  vlastnosti robotov. Či robotov skolí jedna rana (ARES),  alebo tu je aj možnosť nejakého obmedzeného znovuzrodenia (POGAMUT). V bojových  hrách sa tiež ukázalo vhodné umožniť, aby si hráč pred samotným vstupom do sveta mohol tieto vlastnosti upraviť a tým ovplyvniť priebeh súboja. %na to nemam ref
\end{description}
Obrázok \ref{fig:smer} naznačuje smer, v ktorom sa bude práca uberať. Obrázok popisuje jednotlivé koncepty, s ktorými sa môžeme v týchto hrách stretnúť. Ich umiestnenie naznačuje, kde sa tieto koncepty vyskytujú a ako súvisia s Codewars.
\begin{figure}
\centering
\includegraphics[totalheight=0.3\textheight,width=.7\textwidth]{smer}
\caption{Smer uvažovania nad prostredím}
\label{fig:smer}
\end{figure}
