\indent Už od úsvitu ľudstva trápi človeka nieklko hlavných otázok. Jedna z najznámejšich, torú si už iste každý položil, je "čo sa stane, keď..". čiastočne uspokojivú odpoveď mohol dať až nástup informačných technológií. V spojení hĺbkovou analýzou chovania osoby alebo organizmu, evolučnými (genetickými) algoritmami a algortmami pre učenie dávajú simulácie mocný nástroj. Pokiaľ hovoríme priamo o interakcii človeka so svetom presne tak, ako ho vidíme, určite stoji za zmienku projekt Facade pod taktovkou Procedural Arts. Keďže je ale človek ako taký pomerne veľká neznáma, simulácie prebiehajú častejšie s objektami s presne daným typom chovania, materiálmi alebo robotmi. Simulácie sa používajú na zdokonalenie algoritmov a vzhľadom na vrodenú túžbu človeka hrať a víťaziť, aj na získavanie optimálnych stratégií pri rôznych hrách. \\\\
\indent Cieľom tejto práce je implementovať prostredie pre simuláciu bojov robotov s dopredu pripravenými, užívateľom definovanými stratégiami a na nájdených algoritmoch odskúšať funkčnosť a vhodnosť návrhu aplikácie s dôrazom na cieľové použitie. \\\\
\indent Práca je rozdelená do 8 kapitol, kde prvá a posledná sa zaoberajú programom z hladiska užívateľa, ide o zoznámenie a spustenie programu. V druhej kapitole sa detailnejsie rozvedie spôsob a dôvody používania datových štruktúr a ich využitie. V tretej kapitole, ktorá už predpokladá aspoň základné znalosti programovacieho jazyka (ideálne C), sa čitateľ zoznámi s kompletnou architekturou programu a prepojením medzi modulmi (jednotlivými samostatne fungujúcami časťami). Štvrtá kapitola za zaoberá možnosťou spúšťať simuláciu po sieti a piata a šiesta sa zaoberajú vybraním, popisom a ohodnotením vybraných stratégií.
