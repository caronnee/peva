\indent S nástupom informačných techológií sa neoddeliteľne viaže aj vytváranie hier. Zčasti preto, že počítače boli primárne vyvíjané na riešenie problémov matematického charakteru, ktoré boli a doteraz sú často zadávané vo forme pútavých situácií (viz napr. Achileu a korytnačka pre demonštrovanie vlastností limít). Taktiež tomu napomohlo vyvíjanie umelej inteligencie, za prvý pokus sa dá považovať rok 1952, keď v rámci svojej Phd práce vyvinul študent Cambridgu A.S Douglas prvú grafickú hru OXO na dôkaz svojho tvrdenia o interakcii počítača a človeka. A v neposlednej rade taktiež to, že ako pomôcka pri vedeckej práci boli počítače využívané v rámci univerzít, kde sa k nim dostali študenti a ich kreatívne nápady vyvrcholili roku 1961 uvedením na trh hry SpaceWar.
\indent
\indent Cieľom tejto práce je implementovať prostredie pre simuláciu bojov robotov s dopredu pripravenými, užívateľom definovanými stratégiami a na nájdených algoritmoch odskúšať funkčnosť a vhodnosť návrhu aplikácie s dôrazom na cieľové použitie. \\\\
\indent Práca je rozdelená do 8 kapitol, kde prvá a posledná sa zaoberajú programom z hladiska užívateľa, ide o zoznámenie a spustenie programu. V druhej kapitole sa detailnejsie rozvedie spôsob a dôvody používania datových štruktúr a ich využitie. V tretej kapitole, ktorá už predpokladá aspoň základné znalosti programovacieho jazyka (ideálne C), sa čitateľ zoznámi s kompletnou architekturou programu a prepojením medzi modulmi (jednotlivými samostatne fungujúcami časťami). Štvrtá kapitola za zaoberá možnosťou spúšťať simuláciu po sieti a piata a šiesta sa zaoberajú vybraním, popisom a ohodnotením vybraných stratégií.
