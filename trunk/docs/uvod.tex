\indent S nástupom informačných techológií sa neoddeliteľne viaže aj vytváranie hier. Za rok výrobu prvej hry sa považuje dokonca rok 1947, čo je iba 3 roky po uvedení najznámejšieho počítača ENIAC do plnej prevádzky\cite{historia}. Hra bola založená na princípe katodových trubíc. Vykresľované prvky sa  však museli kresliť externe na papier, čo v súvislosti so zábavným priemyslom neznie veľmi kvalitne. Vtedajšia techológia ale jednoducho neumožňovala vykresľovať. Vývoj hier aj tak pokračoval, zčasti preto, že počítače boli primárne vyvíjané na riešenie problémov matematického charakteru, ktoré boli a doteraz sú často zadávané vo forme hier (viz napr. Achilleus a korytnačka pre demonštrovanie vlastností limít, prípad troch rôzne šikovných strelcov a demoštrovanie stratégií v teorií hier). Taktiež tomu napomohlo vyvíjanie umelej inteligencie. Za prvý pokus sa dá považovať rok 1952, keď v rámci svojej PhD práce vyvinul študent Cambridgu A.S. Douglas prvú grafickú hru OXO na dôkaz svojho tvrdenia o interakcii počítača a človeka. V neposlednom rade taktiež to, že ako pomôcka pri vedeckej práci boli počítače využívané v rámci univerzít, kde sa k nim dostali študenti. Ich kreatívne nápady vyvrcholili roku 1961 uvedením hry SpaceWar na trh. \\
\indent Najskôr tieto hry počítali s aktívnou účasťou živých hráčov. Prekážky, ktoré museli prekonávať k úspešnému dokončeniu hry, boli v zásade fyzikálneho rázu (napríklad vyhýbanie sa padajúcim telesám, stavanie blokov, zostrelenie goríl v slávnej hre \emph{Gorillas} vyvíjanej pre MS-DOS apod.) S nástupom komerčných hier, narastajúcim dopytom po dobrodružných, bojových a strategických hrách bolo nutné vyvinúť takých nepriateľov, proti ktorým sa dalo úspešne bojovať. Hráči zisťovali, ako sa daný nepriateľ správa a podľa toho tomu prispôsobovali svoju taktiku. S rozšírením sociálnych sietí ale potreba hráčov hrať hry so živými hráčmi poklesla, pretože u väčšiny ľudí nastane stav nasýtenia spoločnosťou. Preto sa potreba dobrej umelej inteligencie stala dôležitejšou. Zpätný dopad na hráčov spočíval vo zvýšenej analýze chovania nepriateľa a poťazne NPC (NonPlayingCharacters), ktoré sú na strane hráča. Dobrým príkladom potreby analyzovania stratégie proti nepriateľom a so spojencami sú napríklad hry \emph{Baldurs Gate\cite{bioware}}, \emph{Dragon Age\cite{bioware}}, \emph{Mass Effect\cite{bioware}}, kde už je možné nastaviť jednotlivým postavám, za ktoré nehrá hráč, pomerne presnú taktiku, akú majú používať. Ani tento spôsob ale nezabránil občasnému údivu nad tým, čo ktorá postava práve robí. \\
\indent Cieľom tejto práce je implementovať prostredie pre simuláciu bojov robotov s dopredu pripravenými, užívateľom (ďalej hráčom) definovanými stratégiami a na nájdených algoritmoch odskúšať funkčnosť a vhodnosť návrhu aplikácie s dôrazom na realizáciu. Hráč napíše algorimus v zrozumiteľnom, ľahko naučiteľnom jazyku vytvorenom pre účel tejto aplikácie, robot ho bude vykonávať a hráč bude mať možnosť sledovať vývoj situácie graficky.
\indent Prostredie by malo byť súčasne rozšíriteľné a dostatočne univerzálne, aby sa výsledky testovania algoritmov pomocou tohoto prostredia dali použiť na vačšinu bojových, strategických a dobrodružných hier. \\
\indent Bakalárska práca čerpá zo zdrojov uvedených v literatúre, logika spracovania je však pôvodná. Text práce sa skladá z piatich kapitol. Prvá obsahuje úvod do problematiky a dôvod implementácie. Druhá za zaoberá analýzou a návrhom takého to prostredia. Tretia kapitola obsahuje implementačné detaily, náplňou štvrtej je porovnanie s inými, podobnými prostrediami. Posledná, piata kapitola zhŕňa výsledky práce a navrhuje možné zlepšenia.
