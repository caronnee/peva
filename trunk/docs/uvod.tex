\chapter{Úvod}
%chcem napisat, ze sa budem venovat algoritmom chovania robotov
%nieco ako uvod do problematikly, karela herbert
Bakalárska práca sa zaoberá, ako názov naznačuje, súťažením algoritmov. Sútaženie však musí prebiehať zábavnou a zrozumiteľnout formou 
%( tot prec-to aby to nebola jednorazovka, kchceme aby sutazili, ze? zrozumitelne definovanie problemov, ktorym sa budeme venovat, vysledok, aky to ma predstavovat ).
 Vhodným suťažiacim kritériom je napísanie algoritmu chovania robota, ktorý  bojuje s inými robotami o život. \\ % BLE
%Co asi chcem nakodit a preco - ne, je to moc skoro
%NEJAKE OBECNE KECY O ALGORITMOCH CHOVANIA, HERBERT, KAREL, chcem napisat, ze k nieomu to, ze si moze naprogramovat vlastneho robota, vazne bude
%Karel vzbudl zaujem o programovanie
Jedným z prvých programov, ktorý využival koncept programovateľneho robota, je KAREL. Tento program bol vytvorený pre podporu výučby programovacích jazykov. Jednoduchosť jazyka, možnosť pozorovať jednotlivé kroky algoritmu Je tu prilezitost potrapit svoje schopnosti zdanlivo banalnymi prikladmi (napr. Napiste program, po skonceni ktoreho bude robot v strede svojho sveta ) viedla k vytvoreniu sutazi, ktore roznymi obmedzeniami  kladenymi na algoritmus nutili programatora pristupovat k problemu kreativne. \\
%herbert ako aplikacia algoritmu
Dobrým príkladom takejto súťaže je hra HERBERTt\ref{TODO}, kde bolo cieľom, prejsť v neprerušenej postupnosti všetky biele políčka v šachovnicovom svete. Zistťt postupnosť príkazov, ktoré sú riešením, bolo triviálne. Problémom však bolo zapísať ich pomocou rekurzií.
%tu chcem smerovat k tomu, ze chceme, aby sme boli schopni napisat algoritmus a obmedzit svet.
\section{Motivacia}
ko sme naznačili v úvode, vymýšlanie stratégií chovania robota a následné pozorovanie výsledku je dobrým spôsobom ako si overit svoje schopnosti zábavnou a hravou formou. V predstavených hrách je ale jedinym súperom zadaný problém. Súťažiaci tak rátaju s pevne danými dátami ( rozostavenie policiek a podobne ). Na podobnom principe je založené aj súbežne pustenie algoritmov s tým, že ich vykonávanie môže ostatným škodiť. Algoritmus súťažiaceho musí rátať s nerovnakým prostredim sveta a môze dokonca profitovať zo znalosti stratégie súperov. Preto sa v tejto bakalárskej práci zaoberáme algoritmami robotov, ktoré sa budú vykonávat paralelne v rovnakom prostredí. Toto prostredie sa však môže chovaním algoritmov zmeniť, s čim musí algoritmus počitať.\\
%robocode, ako vyzera, a ze je optimalnym
Najbližšim príkladom toho, čo by sme chceli docieliť je hra ROBOCODE\ref{ble}, kde uživateľ programuje v jazyku Java tanky. Tanky hľadajú a ničia nepriateľské tanky.  Vo virtuálnomsvete sú nastavené základné obmedzenia, s ktorymi musí uživateľ počítať.uvazovat.  Napr. čo sa stane, ak strieľa tank príliš často a pod. Toto riešenie je strednou cestou medzi extrémami v podobných programov, ako je sú:
\begin{itemize}
\item ARES  - hra dvoch hráčov. Odohráva sa na mieste simulujúcom pamäť počitača. Úlohou hráča  je naprogramovať robota v tomto prostredi v strojovom kóde (assembleri). Kritériom úspešnosti je program, ktorý sa bude vykonávat ( prakticky hociaký ) a hodnotiacou funkciou je čas, za ktorý program pobeži. Program skonči v okamihu, keď sa pokúsi vykonať neplatnú inštrukciu (napr. delenie nulou, skok na adresu nula, prázdnu inštrukciu ).  Cieľom je  prepisať pamäť takým spôsobom, aby sa druhy program ukončil. Hráč dopredu nepozná ani program protihráča ani dáta, ktorými je inicializovaná pamäť, ak sa hráči dopredu nedohodnú. Oba programy su podobne ako v reálnom svete uložené v pamäti počitača a keďže pamät je zdieľaná, môžu si navzájom prepisovať dáta alebo dokonca inštrukcie. Extrém, ktorý tento koncept prináša, je: %sumarizacia
\begin{itemize}
\item použitie jednorozmerného priestoru ( pole pamäte ) 
\item obmedzenie na počet hráačov ( 2 )
\item nutnosť poznať do hlbky assembler, jazyk, v ktorom je zapíisaný algoritmus
\item možnosť zásahu do algoritmu ostatných hráčov a teda jeho zmena
\item objekty vyskytujuce sa v hre sú iba dáta a inšrukcie
\item hra sa odohráva na najnižšej možnej úrovni -  nie sú tu teáa robotov.
\end{itemize}
\item POGAMUT je už vysoko komplexná hra na robotov. Algoritmy sa dajú programovať v Jave, čim je dovolené použivať špeciálne znaky jazyka, ako je napríklad preťaženie, dedičnost, atd. Prostredie je trojrozmerné a tým majú roboti možnosť širokej škály hybov, skákania po stene, sklony, pohlady hore a dole. Spôsob, akým sa ubližuje ďalšim robotom, je, že intuitívnejší robot vystreľuje obmedzene množstvo striel a ma na výber viac zbraní, ktoré sa líšia presnosťou zásahu. Hrác ma dokoca možnosť ručne riadiť vlastného robota proti naprogramovanému a tým otestovat vhodnosť jeho algoritmu. \\
Extrémom v tomto kontexte sú:
\begin{itemize}
\item trojrozmerný priestor, ktoýy ponúka množsto možností, ako realizovať pohyb - lezenie po stenách, skákanie, graviátacia
item množstvo objektov pôsobiacich na robota, napr. úkryty, strely, vyhýybanie sa strelám, obehnutie prekážky, aplikovanie pathfindingu, obmedzenie
\item zobrazovanie dobre zrozumiteľne pre pozorovateľa
\item možnosť obmedzeného videnia robotov, robot sa môže schovat alebo byť tak ďaleko,  že ho algoritmus nezaregistruje
\end{itemize}
\end{itemize}
Uvedené hry uspokojujúco spľňaju základnú problematiku boja algoritmov. V ARES-e je kritériom zostať nažive podobne ako v ROBOCODE, v POGAMUTe je naviac vopred dane kritické množstvo protivníkov. Hodnotiaca funkcia je rýchlosť, kto skôr splní cieľ, vyhráva.\\
%Zostrojenie takejto hry ale vyžaduje podrobnejši prehľad o nárokoch na jazyk, svet a samotných robotov. Preto sa zameriame na nasledovné charakteristiky týchto hier:
%\begin{description}
%\item[Priestor hier] Kým v ARES-e ide o 1D priestor (jedna veľka pamät - pole), boj vPOGAMUT-e sa odohráva v 3D priestore. S tým súvisí pohyb po ireálnom svete. 3D priestor má omnoho viac možnosti, ako realizovat pohyb. Je nutné zváziť, či bude povolené lietanie, padanie, pohľad zhora, zdola, vrhanie zbraní zboku, a v akých smeroch sa objekty sveta odrážajú a podobne. V ARES-e sa o pohybe, ako ho pozname ( plynulý prechod z miesta A na miesto B ) nedá ani hovoriť, pretože všetky akcie súviace so svetom sú inštrukcie a zmeny v pamati. 
%\item[Kritéria úspešnosti] V ARES-e je jasné, že hra skonči, keď háač nedokáže nadalej vykonávať svoj program . V POGAMUT-e je situácia o poznanie horšia: Pri programovaníi robota sleduje programátor (hráč  dva ciele:
%(a) buď naprogramovať takého robota, ktorého zdolať bude výzva, alebo 
%(b) naopak takého robota, o ktorom sa všeobecne vie, že síce bude poraziteľný, ale nie je ľahké ho obísť. To znamená naprogramovať takého robota, ktorého zdolávať bude výzvou, ale ktorý bude mať chyby, ktorých sa da využiť.
%\item[Spôsoby boja]
%POGAMUT na rozdiel naviac od ARES-a implementuje omnoho viac spôsobov, ako ublížiť robotovi, od rôznych zbraní po odrazenie guliek, rýchleho spadnutia na zem, atd. Jediným spôsobom, ako je možné v ARES-e poškodiť protivníkovi, je. prepísať mu tú časť pamäte, o ktorej je dôvod predpokladať, že ju bude v dohľadnej dobe potrebovať. Vyhodnotenie algoritmu nie je tak možné po častiach, ale až po skončeni celej simulácie. V POGAMUT-e je možné rozlíšiť  už v priebehu simulácie, ako a či robot zasiahol protivníka.
%\item [Vykonávanie programu] Ďaľšim prístupom, ktorý je v vytváraní hry dôležitý, je aj spôsob, v akom poradí sú akcie hráčov interpretované. Aby ostatní hráči neboli znevýhodnení, je vhodné vykonávať jednotlivé časti nezávisle od na ostatných robotov podľa jednotných pravidiel pre všetkých. V ARES-e je to jednoduché, hra prebieha po kolách. Každé kolo znamená vykonanie aktuálnej inštrukcie, čo je to spravodlivé pre všetkých hráov. V POGAMUT-e je nutne zaistiť paralelizáaciu, aby robot nebol závislý na vykonávani programu ostaných robotov.
%\end{description}
\section{Ciele práce}% (ilúzie - tvrdá realita)}
%tu to zmenit na ciel, na ktorom sme sa vraj dohodli - super
{\bf Cieľom bakalárskej práce je umožniť užívateľovi naprogramovať robota, ktorého algoritmus chovania bude súťažit s ostatnými robotmi v prostredi, kde si navzájom roboti môžu ubližovat.}\\ %ciele neskôor
% a potom sa na to odkazovat. Snad je to to, co sakra chcel.
Bakalárska práca sa zaoberá vytvorenim vhodného nástroja, v ktorom môže uživateľ meniť svet, odohrávaju sa súboje, naprogramovať robota a zistťt úspešnosť napísaného algoritmu. Program by mal byť napisaný  tak, aby bol portabilný.\\ 
\newline
Preto je potrebné sa zamerať najmä na: %upresnenie cieľov
\begin{description}
\item [Vytvorenie ireálneho  sveta]
Naprogramovaný robot by mal žit vo virtuálnom isvete, kde je jednoduché sledovať postup vykonavania jeho algoritmu. Z toho vyplýva nárok na prostredie, v ktorom sa bude súboj robotov odohrávať. Súčasťou sveta budú objekty, ktoré interaguju s robotmi a prinášajú tak do vymýšľania stratéegií komplikovanejšie prvky. V ARES-e reprezentujú tieto objekty dáta uložené vo virtuáonom  svete (pamäti), v pripade POGAMUT-a sú to steny, teleporty, priepasti, strely a pod. Treba tiež vymedziť a implementovať také objekty do sveta, ktoré prispievajú k vymýšľaniu sofistikovanejších stratégií. To zahŕňa steny a ich vlastnosti, napr. priehľadnosť, existencia predmetov na dobijanie zdravia, streliva a pod. Predpokladá sa, že tak vytvorený virutálny  svet bude možne upravovať a vytvárať, aby algoritmy boli napisané "na telo" jednej mapy/počiatočnému stavu sveta. Hráč bude mať možnosť ovplyvniť/zmeniť správanie  virtuálneho sveta.\item [Dynamika sveta] 
Naprogramovani roboti budú mať možnosť bojovať,  t.j. si ubližovať a výsledok útoku bude známy v okamihu ublíženia pre ľahšie vyhodnotenie programu. Roboti vo svete sa budú pohybovať všetkými smermi a interagovať s ostatnými objektami vo svete (OK rychlost je uz vlastnost, to by som riesila osobitne JJ. Hrac by mal mat tiez volbu utoku robotov na blizko aj na dialku pre lepsie strategicke moznosti, inak by sa hra zvrtla na "najdi robota a kopni ho".????
\item [Životný cyklus robotov]
Život robotov bude začinať vstupom do sveta a končiť opustenim sveta. Možnosť nejakého znovuzrodenia ako v hre POGAMUT sa nebude pripúšťať, ale bude otázkou ďaľšieho rozširenia. Vítazný robot ostáva živý.  Životný cyklus robota sa bude dať naprogramovať pomocou nejakého programovacieho jazyka, ktorý bude dostatočne zrozumiteľný aj pre laika tatinka)
\item [Vlastnosti robotov] 
Ani jeden z uvedených programov ale nemá možnosť špecifikovať, ake budú  jednotlivé  vlastnosti robotov. Či robotov skolí jedna rana (POGAMUT),  alebo tu je aj možnosť nejakého obmedzeného znovuzrodenia (ARES). V bojových  hrách sa tiež ukázalo vhodne umožniť, aby si hrač pred samotným vstupom do sveta mohol tieto  vlastnosti upravťt a tým ovplyvnil priebeh suboja.
\end{description}
Obrazok na (TODO) naznačuje smer, v ktorom sa bude práca uberať. 
