\indent Už od úsvitu ľudstva trápi človeka nieklko hlavných otázok. Jedna z najznámejšich, torú si už iste každý položil, je "čo sa stane, keď..". čiastočne uspokojivú odpoveď mohol dať až nástup informačných technológií. V spojení hĺbkovou analýzou chovania osoby alebo organizmu, evolučnými (genetickými) algoritmami a algortmami pre učenie dávajú simulácie mocný nástroj. Pokiaľ hovoríme priamo o interakcii človeka so svetom presne tak, ako ho vidíme, určite stoji za zmienku projekt Facade pod taktovkou Procedural Arts. Keďže je ale človek ako taký pomerne veľká neznáma, simulácie prebiehajú častejšie s objektami s presne daným typom chovania, materiálmi alebo robotmi. Simulácie sa používajú na zdokonalenie algoritmov a vzhľadom na vrodenú túžbu človeka hrať a víťaziť, aj na získavanie optimálnych stratégií pri rôznych hrách. \\\\
Cieľom tejto práce je implementovať simuláciu bojov robotov s dopredu pripravenými stratégiami a na nájdených algoritmoch odskúšať funkčnosť a vhodnosť návrhu aplikácie s dôrazom na cieľové použitie. \\\\
Práca je rozdelená do 8 kapitol, kde prvá a posledná sa zaoberajú programom z hladiska uživateľa, ide o zoznámenie a použiteľnosť programu. V druhej kapitole sa detailnejsie rozvedie spôsob a dôvody používania datových štruktúr a och vyuzžitie. V tretej kapitole, ktorá už predpokladá aspon základne znalosti programovacieho jazyka, sa čitateľ zoznámi kompletnou architekturou a prepojením medzi modulmi. Stvrtá kapitola za zaoberá možnosťou spúšťat simuláciu po sieti a piata a šiesta sa zaoberajú vybraním, popisom a ohodnotením vyraných stratégií.
