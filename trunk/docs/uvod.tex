	\indent S nástupom informačných techológií sa neoddeliteľne viaže aj vytváranie hier. Za rok výrobu prvej hry sa považuje dokonca rok 1947, čo je iba 3 roky po uvedení najznámejšieho počítača ENIAC do plnej prevádzky. Hra bola založená na princípe katodových trubíc a vykresľovacie prvky sa museli kresliť externe na papier, čo v súvislosti so zábavným priemyslom neznie veľmi kvalitne, ale vtedajšia techológia jednoducho neumožňovala vykresľovať. Vývoj hier aj tak pokračoval, zčasti preto, že počítače boli primárne vyvíjané na riešenie problémov matematického charakteru, ktoré boli a doteraz sú často zadávané vo forme hier (viz napr. Achilleus a korytnačka pre demonštrovanie vlastností limít, prípad troch rôzne šikovných strelcov a demoštrovanie stratégií v teorií hier ). Taktiež tomu napomohlo vyvíjanie umelej inteligencie. Za prvý pokus sa dá považovať rok 1952, keď v rámci svojej Phd práce vyvinul študent Cambridgu A.S Douglas prvú grafickú hru OXO na dôkaz svojho tvrdenia o interakcii počítača a človeka. A v neposlednej rade taktiež to, že ako pomôcka pri vedeckej práci boli počítače využívané v rámci univerzít, kde sa k nim dostali študenti a ich kreatívne nápady vyvrcholili roku 1961 uvedením hry SpaceWar na trh. \\
\indent Sprvu tieto hry počítali s aktívnou účasťou živých hráčov. Prekážky, ktoré museli prekonávať k úspešnému dokončeniu hry, boli v zásade fyzikálneho rázu ( napríklad vyhýbanie sa padajúcim telesám, stavanie blokov, zostrelenie goríl slávnej hre Gorillas vyvíjanej pre MS-DOS 5 pod. ) S nástupom komerčných hier, narastajúcim dopytom po dobrodružných, bojových a strategických hrách bolo nutné vyvinúť takých nepriateľov, proti ktorým sa dalo úspešne bojovať. Hráči zisťovali, ako sa daný nepriateľ správa a podľa toho tomu prispôsobovali svoju taktiku. S nástupom sociálnych sietí ale potreba hráčov hrať hry so živými hráčmi poklesla, preto sa potreba dobre, uveriteľnej umelej inteligencie stala dôležitejšou. Zpätný dopad na hráčov spočíva vo zvýšenej analýze chovania nepriteľovi a poťazne NPC (NonPlayingCharacters), ktorá sú na strane hráča. Dobrým príkladom analyzovania stratégie proti nepriateľom a so spojencami sú napríklad hry Baldurs Gate, Dragon Age, Mass Effect, kde už je možné nastaviť jednotlivým postavám, za ktoré nehrá hráč, pomerne presnú taktiku, akú majú používať. Ani tento spôsob ale nezabránil občasnému údivu nad tým, čo ktorá postava práve robí. \\
\indent Cieľom tejto práce je implementovať prostredie pre simuláciu bojov robotov s dopredu pripravenými, užívateľom definovanými stratégiami a na nájdených algoritmoch odskúšať funkčnosť a vhodnosť návrhu aplikácie s dôrazom na cieľové použitie. Užívateľ teda napíše algorimus v zrozumiteľnom, ľahko naužiteľnom jazyku vytvorenom pre účel tejto aplikácie, robot ho bude vykonávať a užateľ bude mať možnosť sledovať vývoj situácie\\
%\indent Práca je rozdelená do 8 kapitol, kde prvá a posledná sa zaoberajú programom z hladiska užívateľa, ide o zoznámenie a spustenie programu. V druhej kapitole sa detailnejsie rozvedie spôsob a dôvody používania datových štruktúr a ich využitie. V tretej kapitole, ktorá už predpokladá aspoň základné znalosti programovacieho jazyka (ideálne C), sa čitateľ zoznámi s kompletnou architekturou programu a prepojením medzi modulmi (jednotlivými samostatne fungujúcami časťami). Štvrtá kapitola za zaoberá možnosťou spúšťať simuláciu po sieti a piata a šiesta sa zaoberajú vybraním, popisom a ohodnotením vybraných stratégií.
