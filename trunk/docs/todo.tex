\section{Možné zlepšenia}
Aplikácia Codewars je len základným rozhraním a obsahuje len najzákladnejšie potreby pre vizualizáciu algoritmov. Nasledujúce zlepšenia by zvýšili univerálnosť aplikácie a jej funkčnosť.\\
\begin{itemize}
\item V samotnom priebehu hry by bolo dobré, aby robot, ktorý je na rade, hovoril, ako u neho prebiehajú inštrukcie. Tento spôsob je vhodný pre príjemnejšie debugovanie, ale rozhodne nie je nutný.
\item Možnosť implemetovať kooperujúcich robotov.
\item Škálovateľnost: užívateľ by mal možnosť zadať ďalšie podmienky, kedy robot úspešne skončí svoju misiu, napríklad Visit(x,y,z), VisitSequence(a,b,c), atd, tu by boli tiež prípustné forcykly  a while a pod., ale tieto pravidlá by boli zavazne pre vsetkých účastníkov.
\item Roboti by mohli chciet posielať spravy, nieco ako radio, malo by omedzený dosah, mohli by si zdeľovať informácie 
\item Možnosť poškodiť kód inému robotovi.
\end{itemize}
%V prípade viditelnoti kódu pre všetkých:
%Pre a proti - za hovorí fakt, že je to vhodné pre debugovanie
%Proti - prezrádzal by sa kód neurčeným osobám(protihráčom), ktorí by potom mohli nájsť lepšiu stratégiu nie vlastným pozorovaním, ale zo znalosti kódu, čo nie je úplne zámerom.
%V prípade kódu pre iba jedného:
%za - zvýši sa debugvateľnoť jedného kódu, ale zase keďže sieťová implementácia je robená tak, že hlavný server iba rozposiela informácie o všetkých robotoch na bojisku, musel by súčasne s každou vykonávanou inštrukciou posielať informáciu o tom, že sa vykonala, čo považujem za mrhanie sieťovým potenciálom. druhá možnosť je, že by sieť bola robená tak, že akonáhle príde na radu robot, posle sa informácia dotyčnémi stroju, ten vykoná inštrukcie, výsledky si bude schovávať pre seba a serveru pošle iba výslednú inštrukciu. tento spôsob sa zdá vhodný, ale ešte nie je implementovaný:)
%
%škálovateľnosť:
%Možnosť pre užívateľa samostatne si definovať penlizácie inštrukcií, alebo ich poprípade úpne vypnuť - nastaviť všekoto na 0. Dôvod je simulácia architektur (viz Eurobot a pod.), takze nastavenie inštrukcií. ktoré je potreba k behu + hĺbka rekurzie, veľkost mantisy, veľkosť integeru, prístup k štruktúre typu point, volanie implementovaných funkcií, ako napríklad step(), see a pod.


%zistit, ci sa mi vazne nepokazi, ked inicializujem v kazdom node moznost zanorenia do nejakych 7, potazne 100 + ako to vyzera pri sietovom zatazeni
%povedzme 100 funkcii, kazda 100 premennych definovanych -> 100000 * (sizeOfNode = 12000000 - proste dostatocne malo)
%
%	Zistit, kedy sa mi poze podarit pretiect zasobnik
