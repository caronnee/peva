\section{Analyza vytvarania dialogov v SDL}
SDL nemá samostatné dialógy, ako je tomu napríklad u QT. Z dôvodov vačšej skúsenosti bolo vybrané SDL a preto bolo nutné tieto komponenty naimplementovat. 
\subsection{Analyza funkcie a preddefinovanou hodnotou parametru}
zavedeniu funkcie s preddefinovanou hodnotou parametre má jediný, pomerne malý dôvod. V okamihu, keď sa robot a nejaký iný objekt sa naplánujú na rovnaký čas, ten, ktorý sa naplánoval skôr, skôr tiež príde na radu (jedna z podmienok fair-play). Ak však robot zavolá funkciu, ktorá má preddefinovaný parameter, po zavolaní sa ocitne pred robotom, aj keby mali rovnaké programy a začínali rovnaký čas. To môže tomuto robotovi priniesť isté výhody, pretože sa dostane na radu skôr s predstihom minimálne 3 tickov (1 za assign a 2 za loadovanie premenných). Za ten čas sa podla tabuľky dá spraviť inštrukcia STEP a teda sa napríklad efektívne vyhnúť strele. V skratke, čím skôr sa robot dostane na radu, tým je to pre neho výhodnejsie, lebo môže skôr reagovať na udalosť a predísť následkom a tým viac sa na radu vlastne dostane. Príklad: Strela S je naplánovaná na čas \it{s1}, robot R je po rozmýšľaní 10 tickov pred strelou strelou (predpokladáme, že zavolal funkciu s preddefinovaným parametrom a inak by bol naplánpvaný maximálne 9 tickov pred S) Za ten čas sa môže 2x pohnúť, poťažne použiť inštrukciu see, zistit polohu strely a vystreliť jej smerom.
