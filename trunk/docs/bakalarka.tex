\documentclass[a4paper,11pt,final]{report}
\usepackage[utf8]{inputenc}
\usepackage[slovak]{babel}
\begin{document}
\begin{titlepage}
\begin{center}
\vspace{1.5in}
{\rm Univerzita Karlova v Prahe\\
    Matematicko-fyzikálna fakulta}
\par
\vspace{0.7in}
{\huge \bf Bakalárska práca}
\par
\vspace{0.5in}
{Eva Pešková}
\par
\vspace{0.5in}
Codewars, vojna robotov
\par
\vfill
Katedra softwarového inžinierstva
\par
\vspace{0.5in}
Vedúci bakalárskej práce: Mgr. Tomáš Poch
\par
\vspace{0.5in}
Všeobecná informatika
\par
\vspace{0.5in}
2009
\end{center}
\end{titlepage}
\vfill
Prehlasujem, že som svoju bakalársku prácu napísala samostatne a \\
výhradne s použitím citovaných prameňov. \\
Súhlasím so zapožičiavaním práce.\\
\par
V Prahe \today
\begin{flushright}
Eva Pešková
\end{flushright}
\tableofcontents
\chapter{Uvod}% na com sa to skusa, preco, co je hlanym  cielom, apod.
\chapter{Analýza}
\section{Codewars a jeho alternativy}
\section{Problem optimalnej strategie viacerych hracov}
\section{Prakticke vyuzitie}
\chapter{Implementácia prostredia}
\section{Hracia plocha} % z coho sa sklada
\subsection{Generovanie map}% ako sa generuje, podpora ukladania, zadne zhluky, vyhody, nevyhody, 
\subsection{Vlastnosti stien} % vseobecne oba rozne vlastnosti, ake maju steny, k comu to je
\section{Jazyk hry} %buison, doxygen, preco
\subsection{Priebeh penalizacie za instrukciu}
\subsection{Detekcia zacyklenia}
\section{Vlastnosti robotov}
\section{Sietova komunikacia}
\chapter{Výber a popis testovaných algoritmov}
\section{Algoritmus 1}
\section{Algoritmus 2}
\section{Algoritmus 3}
\chapter{Testovanie}
\chapter{Vylepšenia}
\chapter{Záver}
\newpage
\addcontentsline{toc}{chapter}{\bfseries Literatúra}
\begin{thebibliography}{blablablabla}
\end{thebibliography}
\end{document}
