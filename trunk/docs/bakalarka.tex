\documentclass[12pt,a4paper,notitlepage]{report}

\usepackage[utf8]{inputenc}
\usepackage{slovak}
\usepackage{indentfirst}
\usepackage{longtable}
%%%%%%%%%%%%%%%%%%%%%%%%%%%%%%%%%%%%%%%%%%%%%%%%%%%%%%%%%%%%%%%%%%%%%%
%% custom settings and definitions - begining

\def\CS{$\cal C\kern-.1667em\lower.5ex\hbox{$\cal S$}\kern-.075em $}

%% custom settings and definitions - end
%%%%%%%%%%%%%%%%%%%%%%%%%%%%%%%%%%%%%%%%%%%%%%%%%%%%%%%%%%%%%%%%%%%%%%

\pagestyle{plain}
\frenchspacing 
\usepackage[utf8]{inputenc}
\usepackage{a4wide}
\usepackage{amsmath}
\usepackage{amsfonts}
\usepackage{amssymb}
\usepackage{amsthm}
\usepackage{graphicx}
\usepackage{psfrag}
\usepackage{vmargin}

\begin{document}

\begin{titlepage}
\begin{center}
\vspace{1.5in}
{\rm Univerzita Karlova v Prahe\\
    Matematicko-fyzikálna fakulta}
\par
\vspace{0.7in}
{\huge \bf Bakalárska práca}
\par
\vspace{0.5in}
{Eva Pešková}
\par
\vspace{0.5in}
Codewars, vojna robotov
\par
\vfill
Katedra softwarového inžinierstva
\par
\vspace{0.5in}
Vedúci bakalárskej práce: Mgr. Tomáš Poch
\par
\vspace{0.5in}
Všeobecná informatika
\par
\vspace{0.5in}
2009
\end{center}
\end{titlepage}
\vfill
Prehlasujem, že som svoju bakalársku prácu napísala samostatne a výhradne s~použitím citovaných prameňov.\\
Súhlasím so zapožičiavaním práce.\\
\par
V Prahe \today
\begin{flushright}
Eva Pešková
\end{flushright}

\newtheorem{definicia}{Značenie}

\tableofcontents
\chapter{Úvod}% na com sa to skusa, preco, co je hlanym  cielom, apod.
\input uvod.tex
\chapter{Analýza}
%\input sdl.tex
\input pravidla.tex
\input analyza.tex
\section{Praktické využitie}
Súčasná implementácia poskytuje dosť širokú škálu použitia. Napríklad:\\
	\begin{itemize}
	\item testovanie umelej inteligencie, kde agent s touto umelou inteligenciou bude ovládať iba niekoľko základných príkazov.
	\item program môže simulovať stratégiu robota pri vykonávaní určitej úlohy. Idea vychádza z predmetu Eurobot, kde reálne zostrojený mechanický robot plní dopredu známu úlohu. Vzhľadom na rozdiel medzi aplikovaním stratégie v temer ideálnom prostredí a v praxi, nie je tento program na pre tneto predmet vhodný.
	\item pri istých parametroch môže simulovať pomerne zábavnú a chytľavú hru typu Herbert, ktorá bola náplňou súťaže ImagineCup pod záštitou firmu Microsoft v rokoch 2005-2008, viz\cite{imaginecup}. Táto hra bola neskôr nadšencami reimplementovaná v programovacom jazyku C++ pre zistenie ideálnej stratégie v tomto jednom žpecifickom prípade. %todo zistit spravne roky 
	\item pomocou codewars sa dá simulovať stratégia na spôsob populárnej hry Thief - hľadanie a nasledné zničenie jedného robota ostatnými, alebo hra Capture the flag, teda prebehnutie z miesta na miesto bez toho, aby bol robot zabitý.
	\end{itemize}
\input impl.tex
\input porovnania.tex
%\chapter{Výber a popis testovaných algoritmov}
%\section{Algoritmus 1}
%\section{Algoritmus 2}
%\section{Algoritmus 3}
%\chapter{Testovanie}
\chapter{Záver}
Cieľom bakalárskej práce bolo implementovať také prostredie, ktoré je použiteľné pre širokú škálu požiadaviek, od testovania algoritmov pre nájdenie miest až po algoritmus, ktorý zabezpečí smrť jedného robota pritom sa nedotkne ostatných robotov. Do veľkej miery sa toto podarilo. \\
Najdôležitejšou časťou práce je samotný pribeh simulácie.
 Testované algoritmy ukázali, že 
Aplikácia Codewars bola vyvíjaná tak, aby tieto požiadavky definované v sekcii\ref{poziadavky} rešpektovala.
\input todo.tex
\newpage
\addcontentsline{toc}{chapter}{\bfseries Literatúra}
\begin{thebibliography}{99}
\bibitem{historia}http://cs.wikipedia.org/wiki/Di\%C4\%9Bjiny\_po\%C4\%8D\%C3\%ADta\%C4\%8Dov\%C3\%BDch\_her\_a\_videoher
\bibitem{bioware}www.bioware.com
\bibitem{pogamut}http://artemis.ms.mff.cuni.cz/pogamut/tiki-index.php
\bibitem{robocode} http://robocode.sourceforge.net/
\bibitem{mlaskal} http://ulita.ms.mff.cuni.cz/pub/predn/pp/
\bibitem{trees}
Justin Zobel,Steffen Heinz,Hugh E. Williams:\\
{\it Burst Tries: A Fast, Efficient Data Structure for String Keys} www.cs.mu.oz.au/~jz/fulltext/acmtois02.pdf
\bibitem{vm}
http://java.sun.com/docs/books/jvms/second\_edition/html/Overview.doc.html\#7565
\bibitem{imaginecup}http://imaginecup.com
\bibitem{simtool}http://mpherbert.codeplex.com/
\bibitem{facade}http://www.proceduralarts.com/
\bibitem{quadtree}http://www.kyleschouviller.com/wsuxna/quadtree-source-included/
\bibitem{usedPictures}http://www.bghq.com/fft 
\end{thebibliography}
\end{document}
