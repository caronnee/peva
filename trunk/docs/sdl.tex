\section {Výber knižnice}
Táto práca bola napísaná v jazyku C++ a keďže ten sám o sebe nepodporuje vytváranie grafických aplikácií ( ako má napríklad Java alebo Pascal), bolo treba vybrať si jednu z množstva knižníc spolupracujúcich s C++. Diskutované knižnice spĺňajú základné požiadavky s ohľadom na túto prácu, čo je napríklad multiplatformnosť, podpora sieťových aplikácií, previazanosť s OpenGL a jednoduchosť práce.
\subsection{QT toolkit}
QT je multiplatformná grafická knižnica, ktorá je zameraná skô na aplikácie masívne využivajúce GUI. Vzhľadom na tuto skutočnost je 2D vykresľoanie pomalšie a preto nebola pre túto prácu úplne vhodná.
Pomocou QT sú napísané aplikácie ako napríklad Skype, Google Earth, Opera apod.
\subsection{Allegro}
Grafická knižnica Allegro (Atari Low-Level Game routines) bola vytvorená pre domáci počítač Atari-ST na začiatku 90tych rokoch, odtiaľ názov. Po skončení vývoja tejto platformu však bola prerobená pre DOS, neskôr vznikla samostatná verzie pre Windows a Unix, ktoré sa vo vydaní stable verzie Allegro 4.0 zlúčili do jedného. V súčastnosti je podporovaných viacero operačných systémov vrátane Mac OS, Iris, Solaris, BeOs a pod.
Allegro samo o sebe obsahuje správu udalostí, vykresľovanie textu, funkcie pre kreslenie úsečiek, elíps, robtovanie obrázkov, zoom a pod., dokonca obsahuje aj podporu pre základné GUI, ako je vytváranie dialógov. Podporuje aj softwarové vykresľovanie 3D objektov súčasne dokáže spolupracovat s OpenGL. Vykresľovanie sa deje pomocou priameho prístupu do video-pamäte alebo bufferovania, keď sa jeden z bufferov vykreslí na scénu, zatiaľ čo na ďalšom sa prevádzajú zmeny. Je podporovaný až triple buffering, keď sa v priebehu vykresľovania scény na obrazovku dá pracovat na scéne ďaľšej a v okamihu, keď sa prvá scéha dokreslí, následuje ďalšia v rade. Medzitým sa môže plocha tiež vykresľovať v treťom bufferi. Efektívne sa tým šetrí čas pre vykresľovanie množstva objektov, tento spôsob je najmä teda výhodný pri vykresľovaní množstva objektov, čo však pre túto prácu nie je potrebné.
\indent
Nevýhodou tejto knižnice je, že je ešte pomerne neznáma a podpora pre ňu sa vo viacerých systémoch musí inštalovať osobitne. %TODO! overit, ci nekecam
\subsection{SDL}
SDL je skratka pre Simple Direct Layer, grafickú knižnicu vytvorenú obzvlášť pre tvorbu hier a multimediálnych aplikácií. Ako už naznačuje názov je to jednoduché, nízkoúrovňové API pre prácu s grafikou. Pozostáva z viacerých oddelených modulov, pričom ten základný umožňuje prácu s nekomprimovanými bitmapami, podporuje hardwarovú akceleráciu, mixovanie farieb, dvojitý buffering pre rýchle vykreslenie scény a priamy prístup k jednotlivým pixelom. Ďalšie funkcie, ako je napríklad rotovanie obrázkov (čo napríklad Allegro podporuje okamžite), zoom alebo načítavanie iných formátov obrázky než ne nekomprimovaná bitmapa, je uložene v samostatných modyloch. Napríklad pre vykresľovanie textu je potrebný modul SDL\_ttf, pre načítavanie iných formátov SDL\_image, pre spracovanie obrázkov SDL\_gfx a pod. \newline
\indent
SDL umožňuje jednej aplikácií vytvoriť len jedno okno, do ktorého sa vykresľuje a tomuto oknu sa nedajú pridávať žiadne prvky, ako je to napríklad u Allegra a QT, pozostáva teda výhradne a informácií o jednotlivých pixeloch v štruktúre.
\newline
\indent
Nevýhodou tejto knižnice je, že GUI ani základné grafické operacnie nie su implicitne implementované, preto sa musia naprogramovať osobitne. Vzhľadom ale na veľkú popularitu SDL medzi tvorcami multiplatformných aplikácií ale nie je problém dotyčné, často používané veci získať už naimplementované.
\newline
\newline
\newline
\indent
Pre túto prácu z týchto knižníc boli vhodné zo spomenutých dôvodov len Allegro a SDL a vzhľadom na väčsie skúsenosti s SDL bola vybraná táto knižnica.
%\huge TODO\normalsize {Výhody, nevýhody, citácie zo stránky} \\
%Kedže súčasť tejto práce bolo naimplementovat prostredie pre botov, kde sa predpokladá minimálne množstvo (oproti iným hrám) volieb, formulárov apod., ktoré sa v SDL nepodporuje, nebolo treba siahať k iným knizniciam, ktoré majú podobné vlastnosti ako SDL. \\
%Under Windows, having these libraries simply means downloading the DLL file and placing it in the same folder as your program. Linux users will need to place the files in their lib directory... the sites should have instructions on where to put them exactly. [libsdl.org]
%\huge TODO \normalsize porovnanie knižníc
\newpage
