\section{Codewars  a programy s podobným zameraním}
Codewars bolo inšpirované niekoľkými programami zaoberajúce sa podobnou tematikou. Z tých je najznámejši medzinárodná simulácia tankových bitiev pod menom Robocode\cite{robocode} a rovnakým problémom, hladaním optimálneho algoritmu, sa zaoberá aj projekt Pogamut\cite{pogamut}, ktorý je v podstate obmenou RoboCode v 3D prostredí.\\
\subsection {RoboCode}
RoboCode je aplikácia napísaná a udržovaná v programovacom jazyku Java. Pôvodne mal tento program za úlohu zábavným spôsobom vyučovať programovaciemu jazyk Java, v ktorom je aj napísaný program pre robota. K cieľu hry sa vyjadruli autori nasledujúco:
\begin{quote}Robocode is an easy-to-use robotics battle simulator that runs across all platforms supporting Java 2. You create a robot, put it onto a battlefield, and let it battle to the bitter end against opponent robots created by other developers. Robocode comes with a set of pre-fab opponents to get you started, but once you outgrow them, you can enter your creation against the world's best in one of the leagues being formed worldwide.\\
	Each Robocode participant creates his or her own robot using elements of the Java language, enabling a range of developers -- from rank beginners to advanced hackers -- to participate in the fun. Beginning Java developers can learn the basics: calling API code, reading Javadocs, inheritance, inner classes, event handling, and the like. Advanced developers can tune their programming skill in a global challenge to build the best-of-breed software robot.\end{quote}
Z toho je vidno, že Codewars je pomerne zjednodušená verzia RoboCode. Tiež systém streľby a kolízií je veľmi podobný, ale na rozdiel od RoboCode Codewars ráta aj s obranou ( pre jednoduchosť sa to dá predtaviť ako akési brnenie). Pdobne ako u Codewars agenti RoboCode vykonajú ďalšiu inštrukciu až potom, čo plne dokončili predchádzajúcu. Nie je však povedané koľko času táto akia zaberá ani nie je možné definovať vlastné rýchlosti vykonávania akcií. V Codewars táto možnosť existuje, viz nižšie. %TODO referencia
Dalším rozdielom je, že RoboCode sú zamerané iba na zničenie protivníkov a nedá sa žiadnym spôsobom toto chovanie zmeniť. Codewars ponúka aj možnosť zmeniť cieľ pre každého jednotlivého robota, takže každý môže mať iný cieľ. RoboCode však na rozdiel od Codewars ponúka možnosť tímoveho hrania, kde neexistuje tzv. 'friendly fire', strely od spriatelených bojovníkov neubližujú. ďalej pre pokročilejších robotov je možné nadefinovať akciu, ktorá sa má vykonať akonáhle je splnená jej podmienka, napríklad OnScannedRobot(Event e), taktiež robot tohoto typy má možnosť definovať viacero akcií súčasne, takže sa napríklas môže ohybovať po kružnici. Ďalší potomok triedy AdvancedRobot má zase možnosť zasielať správy. Robocode sa teda dá označiť za komplexne vybavený program so zameraním na prežitie.
\subsection {Pogamut}
Pogamut je projekt vyvíjaný na Karlovej univerzite v rámci predmetu 'Umelé bytosti'. Cieľom je podľa autorov:\\
\begin{quote}It is a platform designed to facilitate creation and debugging of virtual beings called agents. The principal part is IDE. It is a Netbeans plugin that enables user to code a logic of the agent and then debug it and run it in the virtual environment.
\end{quote}
Cieľom je teda naprogramovať logiku pre robota v 3d prostredí v jazyku Java, pričom napríklad príkazy pre pathfinding sú už k dispozícií. V Pogamute nnexistuje možnosť spolupráce, byť to samotná hra, ktorá tvorí prostredie, dovoľuje. Roboti si taktiež nemôžu posielať správy, což je na pre nepriateľskú atmosféru v hre pomerne pochopiteľné. Úlohou je teda napísať program pre chovanie robota v 3D prostredí, takze na rozdiel od CodeWars má robot možnosť aj skákať, plaziť, bežat a pod. Cieľ je oproti predchádzajúcemu trosku iný, v časovom limite treba zničiť určitý počet nepriateľov a súčasne po každom zabití je robot vzkriesený na novom, náhodnom mieste. Codewars pripúšťa zatiaľ iba jednu šancu, v okamihu, ked robota niekto zabije, vypadáva z hry.%TODO obrazky, zisti, ci som nekecala
