\chapter{Implementácia prostredia}
\section{Hracia plocha} % z coho sa sklada
Na začiatku hry sa vygeneruje mapa s veľkostou, ktorú zadá užívatel alebo sa načíta už uložená mapa. Táto pozostáva zo voľných políčok a stien. Steny sú rôzneho typu, viz \ref{kap}. 
\subsection{Generovanie map}% ako sa generuje, podpora ukladania, zadne zhluky, vyhody, nevyhody, 
\section{Jazyk hry a priebeh hry} %bison, preco
Na začiatku hry sa vygeneruje mapa s veľkosťou, ktorú zadá užívateľ, alebo sa načíta už uložená mapa. Táto pozostáva zo voľných políčok a stien. Steny sú rôzneho typu, viz \ref{kap}. Ďalej celú túto koštrukciu nazývam bojisko. V tomto okamihu je hra pripravená na spustenie. V závislosti na veľkosti bojiska sa čaká na odpovedajúce mnoštvo robotov, ktorí sa zapoja do hry. Toto množstvo je defaultne obmedzené iba zdola. Na zdôvodnenie tohoto rozhodnutia je nutné spomenúť základné vlastnosti programu. 
\begin{itemize}
\item Roboti majú za úlohu doraziť na isté miesto označené stenou ExitWall, o ktorom nie je dopredu nič známe. Informácia o tomto cieli je zámerne utajená z toho dôvodu, že ExitWall tam nemusí byť žiadna. Ak by bol totiž garantovaný cieľ, kód robotov by bol prednostne venovaný rôznym vyhľadávacím algoritmom, čímž by sa znížila funkčnosť programu. Teda ak cieľ nie je garantovaný, nastávajú dve možnosti, ako to tom informovať robota:\begin{itemize}
\item za cieľ by sa simulovalo ľubovoľné políčko, ktoré je mimo mapy. V tomto prípade by robot prinajlepšom cyklicky prehľadával celú mapu, čo je v podstate to isté akoby pozíciu vôbec nepoznal.
\item za cieľ by sa vybralo políčko patriace mape. V tom prípade 
\end{itemize}
\item Je zabezpečená spravodlivosť pre robotov. To znamená,
\end{itemize}

v  že pri neprimarane veľkých mapách bude trvať príliš dlho, než sa roboti vobec  
Nastavenie toho, ze roboti so takto obmedzení, a dá prenastaviť v settings.%TODO rozpisat v+setky menu, +co kde robia
\subsection{Generovanie map}% ako sa generuje, podpora ukladania, zadne zhluky, vyhody, nevyhody, 
\section{Jazyk hry a priebeh hry} %bison, preco
\begin{definicia}
Cieľom určenia robota nazývam políčko na mape sveta, kam sa má robot dostať. Teda ak je pozícia robota rovnaká ako pozícia tohoto cieľu, hra končí. Týchto cieľov môže byť viac. V prípade viacerých cieľových oblastí má užívateľ možnosť zvoliť voľbu \it{Vlastný Exit}, kde sa každému robotomi priradí práve jedna cieľová pozícia a to tá najvzdialenesia od jeho výskytu v mape po začatí hry.
Pod vzdialenosťou rozumiem veľkosť obsahu obdĺžnika v vrcholmi cieľom a výskytom robota. %TODO preco?
\end{definicia}

Samotná hra prebieha tak, že roboti vykonávajú užívateľom definovaný kód až do okamžiku, keď dosiahnu cieľ určenia alebo kým zostane na bojisku maximálne jeden robot. Cieľ môže byť pre každého robota rozdielny a je súčasťou mapy sveta. Hra nie je zabezpečená proti nekonečne dlhotrvajúcim cyklom (napríklad keď každý robot stojí na mieste), predpokladá sa v tomto prípade vstup od uživateľa, ktorým hru prerusí.

\begin{definicia}
Stav robota vzhľadom na svet charakterizuje štvorica: (InstructionPointer IP, Hitpoints HP, VisualMemory vm, Direction direct), kde 
\begin{itemize}
\item InstructionPointer je odkaz do štruktúry programu robota hovoriaci, ktorá inštrukcia je na rade
\item HP je nezáporné celé číslo popisujúce životnosť robota
\item VirtualMemory je štruktúra obsahujúca objekty v zornom poli od posledného volania funkcie 'see()', volanej bez parametrov 
\end{itemize}
Stav robota s ohľadom na inštrukcie, ktore vykonáva, charakterizuje (PointerStack PS, ValueStack VS, Memory m) kde
\begin{itemize}
\item PS zoznam pointerov do štruktúry programu
\item VS je zásobník všetkých naloadovaných hodnôt, parametrov  a návratových adries z funkcií
\item Memory je zoznam všetký doteraz definovaných premenných a funkcií
\end{itemize}
\label{StavRobota}
\end{definicia}

\begin{definicia} 
Pod objektami sveta rozumieme všetky prvky, ktoré svet tvoria a zároveň ho z času na čas menia, t.j. v tomto prípade strely, steny, a samotní roboti. Napríklad strela mení svet tým, že sa pohne alebo narazí, istý druh steny sa môže posunúť a pod.
\end{definicia}

\begin{definicia}
Tick je jednotka virtuálneho času sveta. Timeout je počet tickov,ktoré má robot k dispozícii na premýšľanie a tento čas nesmie prekročiť.
\end{definicia}

\begin{definicia} 
Svet je je dvojica $< FrontaUdalosti, Bojisko >$, kde FrontaUdalosti je fronta objektov, ktoré sú na rade, aby komunikovali so svetom. Presný spôsob definovania poradia tejto komunikácie je popísaný neskôr. Objekty v tejto fronte sa počas priebehu boja menia, napríklad pribúdajú a zanikajú strely, ktoré sú podla \ref{defobjekty} tiež objektami.
\end{definicia}

Pre všetky nasledujúc inštrukcie interagujúce so svetom predpokladajme, že svet je v stave $<FrontaUdalosti, Bojisko>$, kde FrontaUdalostí je postupnosť objektov $(robot X, objekt1, objekt2, objekt3,...objektN)$ a Bojisko je dvojrozmerná matica popisujúca, aké objekty sú na akých pozíciach.
Popis jednotlivých inštrukcií a ich vplyv na svet je symbolicky popísaný takto: (bez parameterov, ktoré nemajú vplyv na svet):\\
\newline
step:\begin {itemize}
\item Svet:$ < NewQueue, m.reinsert(X) > $
\item Bot: $ < NewIP, HP, direct> $
\end {itemize}
turnLeft, turnRight \begin{itemize}
\item $Svet:  <NewQueue, m>$
\item $Bot:  < NewIP, Hp, NewDirect)>$
\end{itemize}
turn  \begin{itemize}
\item $Svet:  < NewQueue, m > $
\item $Bot:   < NewIP, Hp, NewDirect)> $
\end {itemize}
see  \begin{itemize}
\item $Svet:  <NewQueue,m> $
\item $ Bot:  < NewIP, Hp, NewDirect)> $
\end {itemize}
shoot \begin {itemize}
\item $ Svet:  < AddQueue, m> $ 
\item $ Bot:  < NewIP, Hp, Direct)>  $
\end {itemize}
\indent
kde NewQueue je \\ $(Objekt_1, Objekt_2, ..., Objekt_K, X, Objekt_{k+1},.., Objekt_N)$, NewIP je nový ukazateľ do programu robota, nie nutn odlišný od predošlého, NewDirect je číslo 1-4 označujúce aktuálne natočenie robota, reinsert je funkcia, ktorá sa pokúsi robotom pohnut v mape, ak sa nedá vyvolá sa ošetrenie kolízií a AddQueue je\\ $(Objekt_1, Objekt_2, ...,Object_{strela},..., Objekt_K, X, Objekt_{k+1},.., Objekt_N)$ (strela bude na rade typicky pred robotom, ktorý j vypustil, pretože ten robot sa preplánuje, ale strela ešte na rade nebola, tak mś vyššiu prioritu.)
Ďalšie inštrukcie vôbec nekomunikujú so svetom, preto sú uvedené oddelene. Tieto inštrukcie robot vykonáva počas svojho premýšľania. Keďze tým robot vobec nepotrebuje vedieť o objektov sveta iných než je on sám, po prevedení takýchto inštrukcií sa meni stav robota podla tabuľky \ref{VnutroBota},kde IP+1 znamená posun v atuálneho pointera v programe robota o jednu inštrukciu ďalej, AddedIP je uloženie aktuálneho pointera na inštrukciu a vyhlásenie iného pointera za aktuálny (typicky začiatok pricedúry alebo funkcie), VariablesAdded je pamať robota rozrastená o premenné danej funkcie alebo procedúry v danom zanorení, ClearVariables je označenie premennych definovaných v tomto bloku za neplatné, LoadByName je uloženie hodnoty volanej premennej na ValueStack, AddedResult je uloženie výsledku aritmetických operácií na ValueStack a DeleteIP je zrušenie aktuálneho pointera na funkciu a obnovenie predchádzajúceho na inštrukiu, pre ktrou bol prerušený. RemoveValue je spracovanie hodnoty zo ValueStacku a teda jej odobranie a odobranie premennej, do ktorej sa priradzuje.

\begin{table}[ht]
\centering
\caption{Vnutorné príkazy robota}
\begin{tabular}{|l|p{5cm}|}
\hline\hline
Inštrukcia & Stav bota po inštrukcii \\
store & $ IP+1,RemoveParameters$\\
aritmeticke+relacne operácie & $IP+1,AddedResult,m$ \\
Call &  $ AddedIP,VS,VariablesAdded$\\
StartBlock & $IP+1,VS,m$\\
EndBlock & $IP+1,VS,ClearVariables$\\
Load & $ IP+1,LoadByName,m$\\
Label & $ IP+1,VS,m$\\
Jump & $ DeleteIP,VS,t$\\
\hline
\end{tabular}
\label{VnutroBota}
\end{table}

\subsection{Vlastnosti stien} % vseobecne oba rozne vlastnosti, ake maju steny, k comu to je
Steny sú objekty sveta, cez ktoré robot nevidí a ktoré môžu napríklad vychýliť dráhu strely. Nie sú však len statické, exituje viacero druhov stien, ktoré budú kominikovať s mapou. Sú to:\\
\begin{itemize}
\item zvláštne políčko Exit, na ktoré sa má robot dostať. Pokiaľ je políčok viac, a nie je difinované inak, robot sa musí dostať do toho najvzdialenejšieho exitu od miesta, kde začínal.
\item MovableWall, touto stenou je možné pohnúť, ak na políčku za za ňou v smere pohybu nič nie je.
\item TrapWall. Táto stena vždy na náhodný počet tikov zmizne a znova sa objaví. V prípade, že sa v okamžiku jej objavenia je na jej políčku nachádza iný objekt, ráta sa to ako kolízia, viz neskôr.%TODO referencia
\item SolidWall, obyčajná stena bez špeciálnych vlastností.
\end {itemize}
Steny sa dajú kombinovať, to znamená, že napríklad východ sa môže pohybovať alebo miznúť. 
\subsection{Viditeľnost robota}
V úvode si užívateľ má možnost nastavit dve premenné, ktoré ovplyvňujú to, ako ďaleko bud robot vidieť. Sú to premenné  'uhol' a 'vzdialenoť'. Tieto premenné sa navzájom ovlyvňujú a teda počet bodov, ktorý sa medzi nich má rozdeliť, je nezávislý na počte bodov u ostatných vlastností robota. Maximálna veľkosť \'vzdialenosti\' je 32, uhol môže byť až $90\,^{\circ}$. Uhol je zadávaný v stupňoch aby sa predišlo načítavanie reálnych hodnôt a tým k strate informácií.
\begin{definicia}
Pod zorným poľom robota rozumieme kruhovú výseč o polomere 'vzdialenosti'.Robot vidí objekt v okamihu, ked vidí stred políčka, na ktorom objekt stojí a úsečka spájajúca tieto dva body nepretína žiadny ďalší objekt.
\end{definicia}
Vzdialenosť a uhol určujú obdĺžnik, ktorý obsahuje presnú polovicu zorného poľa robota.a Tento obdĺžnik môže mať vzhľadom na podmienku o zadávaní vzialenosti a uhla maximálne obsah 32. Pri vytváraní robota sa vygeneruje obdĺžnik a každému políčku sa priradí maska. Nech má obdĺžnik rozmery (x,y) a jeho dolný ľavý vrchol je na pozícií (0,0), čo je v podstate pozícia bota. Táto maska sa potom spočíta tak, že na vygeneruje úsečka spájajúca(0,0) postupne so všetkými stredmi políčok hornom trojuholníku. Každému políčku potom priradíme 32-bitovú masku, kde jedničky sú na pozíciach čísla toho políčka, cez ktoré táto úsečka prechádzala. Druhá polovica je presne symetrická, teda masky tejto polovičky nemusíme počítať zvlášť. Potom na zistenie, ktoré objekt sú viditeľné a ktoré nie, stačí v mape prejst tento obdĺžnik 2x (raz doľava a raz doprava) s počiatočnou maskou nulovou a v prípade, že sa na i-tom políčku zhoduje maska políčka s majou maskou binárne sčítanou s maskou políčka, na to políčko obot vidí. Ak tam je ďalej objekt, nastaví sa v skúšobnej maske na i.tom mieste nula, inak jedna. Ak sa masku nezhodujú, potom robot na toto políčko nevidí a skúšobná maska bude mať na i-tom mieste nulu.
\indent
\subsection{Priebeh penalizácie za inštrukciu}
\begin{definicia}
Penalizáciou za inštrukciu nazývam počet tikov, ktoré robot stratí, ak inštrukciu vykoná. Pre rôzne inštrukcie môže byť rôzna.
\end{definicia}
V okamihu, keď je na rade vo fronte udalostí nejaký robot, začne robot premýšlať, akú ďalšiu akciu by mal so svetom spraviť. Podľa toho, ako dlho mu 'premýšľanie' sa potom zaradí do plánovaných objektov, viz \ref{thinking}. Dĺžka tejto akcie sa ale pre rovnaký počet inštrukcií pre dvoch rôznych botov môže a typicky bude líšiť. Je to spôsobené tým, že každá inštrukcia môže mať rozdielnu penalizáciu za vykonanie. Napríklad samostatné vykoanie inštrukcie 'step' musí byť niekoľko krát rýchlejšie ako povedzme inštrukcia, ktorá naloaduje hodnotu premennej, obzvlášť ak táto premenná už dlho loadovaná nebola. Ďalej inštrukcia, ktorá zavolá užívateľom definovanú funkciu, zaberie viac procerového času vzhladom na to, ćo musí spraviť so zásobníkom robota, a preto je táto skutočnosť zohľadnená. Podobne napríklad inštrukcie pre násobenie a delenie su zlošitejšie ako proste sčitanie a preto sú viac penalizované.\\
\indent
Robotovi môže týmto spôsobom samozrejme vypršať čas. V takom prípade, ako je naznačené na obrázku \ref{thinking} sa robot vzdá svojej možnosti interagovat so svetom a preplánuje sa. Preplánovanie prebieha tým spôsobom, že sa spočíta, kolko svojho času robot prehmrhal na myslenie, presne o toľko sa posunie v pomyselnejcasovej osi a zaradí sa vo fronte udalostí za posledného bota s rovnakým časom. Samotné inštrukcie majú penalizáciu síce nenulovú, ale najmenšiu, čo odpovedá tomu, že je to pre bota prirodzená inštrukcia a zároveň sa tým zamedzuje tomu, aby robot v prípade samých takýchto inštrukcií nepustil na radu ostatné objekty. Objekty iné ako robot majú rovnako penalizáciu, ale vždy menšiu ako je timeout. Každý takýto objekt má ale už iba funkciu interagujúcu so svetom, takze sa po prevedení tejto akcie zase preplánujú.


\subsection{Detekcia zacyklenia}

\section{Vlastnosti robotov}
Výsledok programu, ktorého inštrukcie robot vykonáva, je závislý na konkrétnom type robota. Typ robota je definovaný niekoľkými parametrami, ktoré si bežný užívateľ samostatne na začiatku hry definuje == UholViditelnost, PolomerViditelnosti, Obrana, Hitpoints, TypStrely, VelkostPamate==
\begin{table}[ht]
\caption{Vlastnosti robota}   % title of Table
\centering                          % used for centering table
\begin{tabular}{cl}            % centered columns (4 columns)
%\hline\hline                        %inserts double horizontal lines
Vlastnost & Vplyv \\   % inserts table
%heading
%\hline                              % inserts single horizontal line
polomer viditeľnosti & Robot dokáže na určitú vzdialenosť rozpoznať object, táto vlastnosť určuje, koľko políčok dopredu v priamom smere( v smere, akým je bot aktualne otočený) vidí. Tento počet sa pod iným uhlom o niečo zmení,\\
uhol viditeľnosti & maximálny rozptyl viditeľnosti - robot nevidí celý kruh okolo seba, ale iba určitú výseč \\
Obrana & Obranne cislo bota. Pouziva sa pri strete s nejakou prekazkou. Viz kolizie. \\
Hitpoints  & životnost bota, akonahle klesne na nulu, z fronty akcií sa jeho nasledujúca akcia odstráni a robot mizne do prepadlišťa dejin\\
Typ strely & Strela je sa definuje podobne ako samotný bot, ale ma iné parametre\\
\end{tabular}
\label{table:vlastnosti}          % is used to refer this table in the text
\end{table}

\begin{table}[h]
\caption{Vlastnosti Strely}   % title of Table
\centering                          % used for centering table
\begin{tabular}{cc}            % centered columns (4 columns)
\hline\hline                        %inserts double horizontal lines
Vlastnost & Vplyv \\   % inserts table
%heading
\hline                              % inserts single horizontal line
Odrážavost & definuje, ako moc sa strela od steny odrazí, v podstate je to to iste ako hmotnosť\\ \hline
Rýchlosť & definuje preplanovanie vo fronte akcií\\ \hline
Hitpoints  & dolet strely, ten sa znižuje s poctom tikov a súčasne kolíziami\\ \hline
Utok & Utocne cislo bota, pouziva sa pri utok, viz kolizie \\ \hline
\hline                              %inserts single line
\end{tabular}
\end{table}

Užívateľ si typ strely definuje tak, že dostane k dispozícií určitź počet bodv a tie musí medzi jednotlive volby rozdeliť\\
Bot premýšla tak, že sa nechá bežat dovtedy, kým z jeho program nenarazí na inštrukciu, ktorá by ovplyvnila svet, alebo kým neprsiahne timeout. Ak presiahne timeout, preruší sa, preplánuje vo fronte udalostí ďalej (presnejšie o TIMEOUT tikov ďalej)\\
Velmi špeciálny je prípad, keď užívateľ nenastaví robotovi žiadnu pamať. Takýto robot nie je schopný si zapatať jedinú premennú a ani vracat návratové hodnoty, teda v kóde programu by nemali byť definované žiadne funkcie, ale iba procedúry.Je nutné povedať, že funkcie a procedúry sa v žiadnom prípade nechovajú ako premenné a teda nezaberajú robotovi pamať. Teda v tomto prípade dojde k masívnemu využitiu rekurzií. Samotne vstavané príkazy sú však vlastne funkcie, ale kedže sa nikam nemajú priradiť, ich návratová hodnota sa zahodí. Nevýhodou tohoto typu bota je však skutočnost, že nemá povolené napriklad strieľanie, pretože to je funkcia, ktorá potrebuje za každých okolností parametre.
