\section{Úvod do Codewars}
V tejto sekcii by som chcela objasniť úlohu a pravidla simulácie Codewars, keďže sa na ne neskôr budem odkazovať. \\ 
Ako názov naznačuje, Codewars je súboj hlavne kódov, teda algoritmov. To sa prejaví konkrétne tak, že každý hráč dostane po spustení minimálne jedného virtuálneho robota na hranie, dalších si neskôr môže pridať. Pre jednoduchosť predpokladajme, že je to práve jeden robot a že máme viacero hráčov, každý s práve jedným robotom. Úlohou hráčov je napísať pre každého svojho robota vlastnú logiku, ktorou sa bude riadiť. Po napísaní tejto logiky v špeciálnom jazyku stvoreným pre Codewars sú títo roboti vypustení do dopredu pripraveného prostredia. \\
\indent Okrem robotov sa v Codewars vyskytujú aj ďalšie objekty, na ktoré má hráč pomerne malý vplyv. V súčastnosti sú to strely a steny. Vplyv robota ( a teda) na tieto objekty bude vysvetlený neskôr pri samotnom popisovaní vlstností a priebehu hry \\%TODO  referencia na tu kapitolu
\indent Robot je vlastne iba postava, ktorou hráč môže pred samotným bojom pripraviť. Príprava spočíva okrem vybavenia ho vnútornou logikou aj v úprave vlastností robota a jeho zbrane. Tieto vlastnosti sa rozdeľujú do určitých tematicky podobných skupín (sekcií), v rámci ktorých sa môžu vlastnosti meniť. Sú to tieto:
\begin{itemize}
\item Vlastnosti pojednávajúce o stave robota
\begin{itemize} 
\item Dĺžka života : je číslo označujúce životaschopnosť robota. Štandartne sa znižuje v okamihu, keď na miesto, kde stojí robot, chce pristúpiť iný objekt. Existuje možnosť, že sa v tomto prípade toto číslo aj tak nezmení, viz \ref{kolizie} 
\item Polomer viditeľnosti: určuje vzdialenosť, do akej maximálne robot môže robot rozoznávať ostatné objekty.
\item Uhol viditeľnosti: určuje kruhovú výseč, vnútri ktorej robot rozoznáva objekty.
\item Veľkosť pamäte : určuje maximálny počet premenných, ktoré si robot pamatá a tým obmedzuje napríklad hĺbku zarekurzenia, keďže každá funkcia má nejaký návratový typ, ktorý si robot musí pamätať
\end{itemize}
\begin{itemize}
\item Obranné číslo: určuje odolnosť robota vočí iným objektom
\item Útočné číslo: určuje útočnosť robotovho tela pri strete s inými objektami
\end{itemize}
\item Vlasnosti týkajúce sa akcieschopnosti na diaľku
\begin{itemize}
\item Ďalekonosnosť zbrane: určuje, ako daleko je možné náboj vystreliť. Ak zbraň prejde viac ako daný počet políčok, automaticky zaniká. Dá sa tak zameniť so životnosťou zbrane
\item Útočnosť zbrane: určuje ako moc je zbraň schopná ublížiť objektom, viz \ref{kolizie}
\end{itemize}
\end{itemize}
Toto rozdelenie znamená, že hráč môže napríklad zvýšit robotovu odolnosť voči stretu s inúmi objektami, ale súčasne mu potom zostáva menej na zvýšenia životaschopnosti robota. Počet bodov, ktoré sa v rámci sekcie prerozdeľujú, je vždy v každom oddieli pre každého hráča rovnaký (T.j. každý hráč prerozdelí X bodov v rámci prvej sekci a  Y v rámci druhej a Z v rámci tretej, kde X, Y, Z nemusia byť nutne navzájom rôzne čísla). Rozdelenie vlastností do sekcií nemá väčší než estetický význam. Predstava, ze na úkor očí bude puška alebo podobný nástroj lepšie strielať, nie je príliš intuitívna a užívateľa by to mohlo pomerne zmiasť. Čo sa týka samotných čísel X a Y, Z, vzhľadom na implementáciu kolízií a následné vypočítavanie poškodenia obete alebo útočníka sú v aktuálnej implementácií tieto dve čísla konštantami o hodnotách 200,50,50. Tieto čísla boli vybrané pre pomerne ľahké zapamätanie a súčasne nie sú moc veľké, aby sa simulácia príliš preťahovala.
\\Hlavnou úlohou hráča je však napísať algoritmus, podľa ktorého sa bude robot správať. Optimálny algoritmus spraví to, robot ako jediný prežije, t.j väčšinu svojich nepriateľov zničí alebo ich prinúti zničíť sa(napríklad navedením na vlastnú strelu alebo podobne).

\subsection{Vlastnosti stien} % vseobecne oba rozne vlastnosti, ake maju steny, k comu to je
Steny sú objekty sveta, cez ktoré robot nevidí a ktoré môžu napríklad vychýliť dráhu strely. Nie sú však len statické, exituje viacero druhov stien, ktoré budú komunikovať s mapou. Sú to:\\
\begin{itemize}
\item Zvláštne políčko Start, ktoré sa používa iba pri generovaní mapy, označúje políčko, kde majú roboti na začiatku simulácie stáť 
\item MovableWall, touto stenou je možné pohnúť, ak na políčku za za ňou v smere pohybu nič nie je.
\item TrapWall. Táto stena vždy na náhodný počet tikov zmizne a znova sa objaví. V prípade, že sa v okamžiku jej objavenia je na jej políčku nachádza iný objekt, ráta sa to ako kolízia, viz neskôr.%TODO referencia
\item SolidWall, obyčajná stena bez špeciálnych vlastností.
\end {itemize}
Steny sa dajú kombinovať, to znamená, že napríklad stena sa môže pohybovať alebo miznúť. 

\subsection{Kolízie}\label{kolizie}
Kolízia je stav, keď na jedno miesto sa chcú dostať dva objekty. Miesto v našom príprade je jedno políčko mapy. V prípade kolízie jeden alebo druhý objekt utrpí a to buď útočník, ktorého útočnosť bola menšia ako obranyschopnosť brániaceho sa objektu(teda neprenikol obranou, akovýsledok sa mu zmení smer, ktorým sa doteraz pohyboval), alebo utrpí brániaci sa objekt, ak jeho obrana nebola dostatočná. V tom prípade dostane tento objekt odpovedajúce zranenie. To, ako sa po úspešnom útoku zachová útočník, závisí na jeho type. Ak je to napríklad strela, zanikne, ak je to stena, svojím pohybom útoči ďalej, ak je to hráč, závisí na jeho algoritme.

\subsection {Priebeh hry}
Hra sa začína vhodením robotov do prostredia v poradí, v akom boli pridaní a roboti by mali začať vykonávať svoj program súčasne. To implementačne znamená, že buď každý nový robot dostane vlastné vlákno, alebo sa určí spôsob zoradenia robotov tak, aby nesimultánnosť bola vidieť čo najmenej. Spôsob preplánovania jednotlivého súčastí CodeWars je popísaný neskôr, nateraz stačí povedať, že každý jednotlivý kus kódu programu robota zaberá istý čas (napríklad na premiestnenie sa alebo vypočítanie premennej), a teda sa robot môže dostať opať na radu až v okamihu, keď tento čas pominie. \\ %TODO
